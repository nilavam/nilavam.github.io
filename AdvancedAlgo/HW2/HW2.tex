\documentclass[12pt]{amsart}
\usepackage[letterpaper, portrait, left = 1in, right = 1in, top = 1.2in, bottom=1.5in]{geometry} 
%\usepackage{setspace} \doublespacing
%\usepackage[letterpaper, portrait, margin=1.3in]{geometry}
\usepackage[table,xcdraw]{xcolor}
\usepackage{amssymb}
\usepackage{amsfonts}
\usepackage{longtable}
\usepackage{amsmath,amsthm}
\usepackage{enumitem}
\usepackage[utf8]{inputenc}
\usepackage{mathtools}
\usepackage{graphicx}
\usepackage{parskip}
\usepackage{multicol}
\usepackage{listings}
\usepackage[skip=0.25pt]{caption}
\usepackage[mathscr]{euscript}
\usepackage{quiver}
\setlength{\parindent}{0pt}
\usepackage{thm-restate}
\definecolor{vividburgundy}{rgb}{0.62, 0.11, 0.21}
\usepackage[driverfallback=hypertex,pagebackref=false,colorlinks,citecolor=vividburgundy]{hyperref}
\usepackage[capitalize]{cleveref}
%\usepackage[cmintegrals,cmbraces]{newtxmath}
%\usepackage{ebgaramond-maths}

%\usepackage{fourier}
%----------FONT OPTIONS----------
% sans-serif
%\usepackage[sfdefault]{FiraSans}
 %\usepackage[sfdefault]{roboto}
% \usepackage[sfdefault]{noto-sans}
%\usepackage[default]{sourcesanspro}

% serif
%\usepackage{CormorantGaramond}

%\usepackage{charter}
\usepackage[T1]{fontenc}
\usepackage{cleveref}
\definecolor{dg}{RGB}{10, 100, 10}
\setlength{\parindent}{0in}
\renewcommand{\qed}{$\hfill\blacksquare$}
% \newtheoremstyle{style}{2pt}{1pt}{\normalfont}{}{\bfseries}{\\}{0cm}{}
% \theoremstyle{style}
\newtheorem{lemma}{Lemma}[section]
% \newtheorem{lemma}{Lemma}
\newtheorem{thm}[lemma]{Theorem}
\newtheorem{prop}[lemma]{Proposition}
\newtheorem{cor}[lemma]{Corollary}
\newtheorem{conj}[lemma]{Conjecture}
\newtheorem{cl}[lemma]{Claim}
\newtheorem{rmk}{Remark}
\newtheorem{defn}[lemma]{Definition}
\newtheorem{qs}{Question}
\newtheoremstyle{styleS}{}{}{\color{dg}}{}{\color{dg}\bfseries}{. }{0cm}{}
\theoremstyle{styleS}
\newtheorem*{sol}{Solution}
\newtheoremstyle{style1}{}{}{\normalfont}{}{\bfseries}{. }{0cm}{}
\theoremstyle{style1}
\newtheorem{prob}{Problem}[section]
\newtheorem*{prb}{Problem}
\newtheoremstyle{style2}{1pt}{4pt}{\normalfont}{}{\itshape}{. }{0cm}{}
\theoremstyle{style2}
\newtheorem{ex}[lemma]{Example}
\newtheorem*{pf}{Proof}

\newcommand{\norm}[2]{
\left\lVert #1 \right\rVert_{#2}
}
\usepackage{mathrsfs}
%\usepackage[table,xcdraw]{xcolor}
\usepackage{booktabs}
\usepackage{tikz}
\usetikzlibrary{matrix}
\renewcommand{\l}{\ell}


\newcommand{\fA}{{\mathfrak{A}}}   \newcommand{\fB}{{\mathfrak{B}}}
\newcommand{\fC}{{\mathfrak{C}}}   \newcommand{\fD}{{\mathfrak{D}}}
\newcommand{\fE}{{\mathfrak{E}}}   \newcommand{\fF}{{\mathfrak{F}}}
\newcommand{\fG}{{\mathfrak{G}}}   \newcommand{\fH}{{\mathfrak{H}}}
\newcommand{\fI}{{\mathfrak{I}}}   \newcommand{\fJ}{{\mathfrak{J}}}
\newcommand{\fK}{{\mathfrak{K}}}   \newcommand{\fL}{{\mathfrak{L}}}
\newcommand{\fM}{{\mathfrak{M}}}   \newcommand{\fN}{{\mathfrak{N}}}
\newcommand{\fO}{{\mathfrak{O}}}   \newcommand{\fP}{{\mathfrak{P}}}
\newcommand{\fQ}{{\mathfrak{Q}}}   \newcommand{\fR}{{\mathfrak{R}}}
\newcommand{\fS}{{\mathfrak{S}}}   \newcommand{\fT}{{\mathfrak{T}}}
\newcommand{\fU}{{\mathfrak{U}}}   \newcommand{\fV}{{\mathfrak{V}}}
\newcommand{\fW}{{\mathfrak{W}}}   \newcommand{\fX}{{\mathfrak{X}}}
\newcommand{\fY}{{\mathfrak{Y}}}   \newcommand{\fZ}{{\mathfrak{Z}}}

\newcommand{\cA}{{\mathcal{A}}}   \newcommand{\cB}{{\mathcal{B}}}
\newcommand{\cC}{{\mathcal{C}}}   \newcommand{\cD}{{\mathcal{D}}}
\newcommand{\cE}{{\mathcal{E}}}   \newcommand{\cF}{{\mathcal{F}}}
\newcommand{\cG}{{\mathcal{G}}}   \newcommand{\cH}{{\mathcal{H}}}
\newcommand{\cI}{{\mathcal{I}}}   \newcommand{\cJ}{{\mathcal{J}}}
\newcommand{\cK}{{\mathcal{K}}}   \newcommand{\cL}{{\mathcal{L}}}
\newcommand{\cM}{{\mathcal{M}}}   \newcommand{\cN}{{\mathcal{N}}}
\newcommand{\cO}{{\mathcal{O}}}   \newcommand{\cP}{{\mathcal{P}}}
\newcommand{\cQ}{{\mathcal{Q}}}   \newcommand{\cR}{{\mathcal{R}}}
\newcommand{\cS}{{\mathcal{S}}}   \newcommand{\cT}{{\mathcal{T}}}
\newcommand{\cU}{{\mathcal{U}}}   \newcommand{\cV}{{\mathcal{V}}}
\newcommand{\cW}{{\mathcal{W}}}   \newcommand{\cX}{{\mathcal{X}}}
\newcommand{\cY}{{\mathcal{Y}}}   \newcommand{\cZ}{{\mathcal{Z}}}

\newcommand{\sA}{{\mathscr{A}}}   \newcommand{\sB}{{\mathscr{B}}}
\newcommand{\sC}{{\mathscr{C}}}   \newcommand{\sD}{{\mathscr{D}}}
\newcommand{\sE}{{\mathscr{E}}}   \newcommand{\sF}{{\mathscr{F}}}
\newcommand{\sG}{{\mathscr{G}}}   \newcommand{\sH}{{\mathscr{H}}}
\newcommand{\sI}{{\mathscr{I}}}   \newcommand{\sJ}{{\mathscr{J}}}
\newcommand{\sK}{{\mathscr{K}}}   \newcommand{\sL}{{\mathscr{L}}}
\newcommand{\sM}{{\mathscr{M}}}   \newcommand{\sN}{{\mathscr{N}}}
\newcommand{\sO}{{\mathscr{O}}}   \newcommand{\sP}{{\mathscr{P}}}
\newcommand{\sQ}{{\mathscr{Q}}}   \newcommand{\sR}{{\mathscr{R}}}
\newcommand{\sS}{{\mathscr{S}}}   \newcommand{\sT}{{\mathscr{T}}}
\newcommand{\sU}{{\mathscr{U}}}   \newcommand{\sV}{{\mathscr{V}}}
\newcommand{\sW}{{\mathscr{W}}}   \newcommand{\sX}{{\mathscr{X}}}
\newcommand{\sY}{{\mathscr{Y}}}   \newcommand{\sZ}{{\mathscr{Z}}}

\newcommand{\ta}{{\tilde{a}}}   \newcommand{\tb}{{\tilde{b}}}
\newcommand{\tc}{{\tilde{c}}}   \newcommand{\td}{{\tilde{d}}}
\newcommand{\te}{{\tilde{e}}}   \newcommand{\tf}{{\tilde{f}}}
\newcommand{\tg}{{\tilde{g}}}   
\newcommand{\ti}{{\tilde{i}}}   \newcommand{\tj}{{\tilde{j}}}
\newcommand{\tk}{{\tilde{k}}}   \newcommand{\tl}{{\tilde{l}}}
\newcommand{\tm}{{\tilde{m}}}   \newcommand{\tn}{{\tilde{n}}}
		         	\newcommand{\tp}{{\tilde{p}}}
\newcommand{\tq}{{\tilde{q}}}   \newcommand{\tr}{{\tilde{r}}}
\newcommand{\ts}{{\tilde{s}}}   
\newcommand{\tu}{{\tilde{u}}}   \newcommand{\tv}{{\tilde{v}}}
\newcommand{\tw}{{\tilde{w}}}   \newcommand{\tx}{{\tilde{x}}}
\newcommand{\ty}{{\tilde{y}}}   \newcommand{\tz}{{\tilde{z}}}

\newcommand{\red}{{\color{red}red}}
\newcommand{\blue}{{\color{blue}blue}}

\newcommand{\into}{\hookrightarrow}
\newcommand{\onto}{\twoheadrightarrow}
\newcommand\N{\ensuremath{\mathbb{N}}}

%\newcommand\L{\ensuremath{\mathbb{L}}}
\newcommand{\bP}{\mathbb{P}}
\newcommand\M{\ensuremath{\mathbb{M}}}
\newcommand\R{\ensuremath{\mathbb{R}}}
\newcommand\Z{\ensuremath{\mathbb{Z}}}
\renewcommand\O{\ensuremath{\emptyset}}
\newcommand\Q{\ensuremath{\mathbb{Q}}}
\newcommand\C{\ensuremath{\mathbb{C}}}
\newcommand{\K}{\ensuremath{\mathbb{K}}}
\newcommand\F{\ensuremath{\mathbb{F}}}
\newcommand{\aff}{\ensuremath{\mathbb{A}}}
\newcommand{\proj}{\ensuremath{\mathbb{P}}}
\newcommand{\dd}{\mathrm{d}}
\newcommand{\m}{\ensuremath{\mathfrak{m}}}
\newcommand{\p}{\ensuremath{\mathfrak{p}}}
\newcommand{\n}{\ensuremath{\mathfrak{n}}}
\renewcommand{\phi}{\varphi}
\renewcommand{\qedsymbol}{\ensuremath{\blacksquare}}
%\newcommand{\st}{\;|\;}
\newcommand{\st}{%
  \nonscript\;
  \ifnum\currentgrouptype=16
    \;\middle|\;
  \else
    \;|\;
  \fi
  \nonscript\;}
\newcommand{\ltr}{\par \noindent \framebox[1\width]{ $\implies$ } \hspace{.2cm}}
\newcommand{\rtl}{\par \noindent \framebox[1\width]{ $\impliedby$ } \hspace{.2cm} }
\newcommand{\abs}[1]{\left| #1 \right|}
\newcommand{\inner}[2]{\left\langle #1, #2 \right\rangle}
\newcommand{\E}[1]{\mathbb E\left[ #1 \right]}
\newcommand{\e}[1]{\exp\left( #1 \right)}
\renewcommand{\P}[1]{\mathbb P\left[ #1 \right]}
\newcommand{\Var}[1]{\text{Var}\left[ #1 \right]}
\newcommand*\circled[1]{\tikz[baseline=(char.base)]{
            \node[shape=circle,draw,inner sep=2pt] (char) {#1};}}
\newcommand{\ds}{\displaystyle}

\DeclareMathOperator{\sym}{Sym}
\DeclareMathOperator{\mds}{MDS}
\DeclareMathOperator{\Tor}{Tor}
\DeclareMathOperator{\Ext}{Ext}
\DeclareMathOperator{\adj}{adj}
\DeclareMathOperator{\Tr}{Tr}
\DeclareMathOperator{\GL}{GL}
%\DeclareMathOperator{\Tr}{Tr}
\DeclareMathOperator{\orbit}{Or}
\DeclareMathOperator{\stab}{Stab}
\DeclareMathOperator{\fix}{Fix}
\DeclareMathOperator{\re}{Re}
\DeclareMathOperator{\im}{Im}
\DeclareMathOperator{\ord}{Ord}
\DeclareMathOperator{\mspec}{mSpec}
\DeclareMathOperator{\spec}{Spec}
\DeclareMathOperator{\frob}{Frob}
\DeclareMathOperator{\id}{Id}
\DeclareMathOperator{\colim}{colim}
\DeclareMathOperator{\loc}{loc}
\DeclareMathOperator{\res}{Res}
\DeclareMathOperator{\rad}{rad}
\DeclareMathOperator{\Res}{Res}
\DeclareMathOperator{\diam}{diam}
\DeclareMathOperator{\arcsec}{arcsec}
\DeclareMathOperator{\arccot}{arccot}
\DeclareMathOperator{\len}{len}
\DeclareMathOperator{\area}{area}
\DeclareMathOperator{\vol}{vol}
\DeclareMathOperator{\ev}{ev}
\DeclareMathOperator{\sgn}{sgn}
\DeclareMathOperator{\supp}{supp}
\DeclareMathOperator{\diff}{d}
\DeclareMathOperator{\Dom}{Dom}
\DeclareMathOperator{\rk}{rank}
\renewcommand{\d}{\diff}
\let\oldend\endlinechar
\renewcommand{\endlinechar}{\oldend}
\newcommand{\open}{\underset{\text{open}}{\subset}}
\newcommand{\divides}{\mathbin{|}}
\newcommand{\set}[1]{\ensuremath{\left\{#1\right\}}}
\newcommand{\sett}{\coloneqq}
\newcommand*\isomap{%
  \xrightarrow{\raisebox{-0.9ex}[0ex][0ex]{$\sim$}}%
}
\renewcommand{\epsilon}{\varepsilon}
\newcommand{\fa}{~\forall~}
%\usepackage[nobottomtitles*]{titlesec}
\usepackage{titletoc}
%\titleformat{\section}[runin]
%{\normalfont\Large\bfseries}
%{}{0pt}{}%
%[\ifthenelse{\equal{\thesection}{0}}{\\\vspace*{0pt}}{\space\thesection}]
\newcommand{\sint}{\sin\theta}
\newcommand{\cost}{\cos\theta}
\newcommand{\tant}{\tan\theta}
\newcommand{\lb}{\left[}
\newcommand{\rb}{\right]}
\newcommand{\lp}{\left(}
\newcommand{\rp}{\right)}
\newcommand{\br}[1]{\lb#1\rb}
\newcommand{\pa}[1]{\lp#1\rp}
\usepackage{pdfpages}
\usepackage{fancyhdr}
	\pagestyle{fancyplain}
	\fancyhf{}
	\fancyhead[C]{\thepage}

%\fontfamily{qcr}\selectfont 
%\usepackage[backend=bibtex]{biblatex}
\usepackage[
backend=biber,
style=alphabetic,%firstinits,
citestyle=ieee-alphabetic,
%natbib=true,
%uniquelist=false,
maxnames=10,
sorting=ynt
]{biblatex}
%\addbibresource{writeup/article/refs.bib}
%\title{\vspace{-1cm}}
\title{\textbf{ADVANCED ALGORITHM DESIGN}\\ Homework $2$}
\usepackage{quiver}
\usepackage[nobottomtitles*]{titlesec}
\usepackage{titletoc}
\titleformat{\section}[runin]
  {\normalfont\Large\bfseries}
  {}{0pt}{}%
  [\ifthenelse{\equal{\thesection}{0}}{\\\vspace*{0pt}}{\space\thesection}]
%\author{{\Large NILAVA METYA} \\ 
%\href{mailto:nilava.metya@rutgers.edu}{nilava.metya@rutgers.edu}\\
%\href{mailto:nm8188@princeton.edu}{nm8188@princeton.edu}}

\date{\vspace{-0.7in}October $27$, $2024$}
\newcommand{\pb}{\section{Problem}~\par}
\newcommand{\soln}{\subsection*{Solution}}
\usepackage{pdfpages}
\usepackage{fancyhdr}
	\pagestyle{fancyplain}
	\fancyhf{}
	\fancyhead[R]{\thepage}
\newcommand{\fa}{~\forall~}
\begin{document}

\maketitle


\pb




\soln








\newpage
\pb

The maximum cut problem asks us to cluster the nodes of a graph $G = (V,E)$ into
two disjoint sets $X,Y$ so as to maximize the number of edges between these sets:
$$\max_{X,Y} \sum_{(i,j)\in E}\pmb 1\left[(i\in X, j\notin X) \lor (i\in Y, j\notin Y)\right].$$
Consider instead clustering the nodes into three disjoint sets $X,Y,Z$. Our goal is to maximize the number of edges between different sets:
$$\max_{X,Y,Z} \sum_{(i,j)\in E}\pmb 1\left[(i\in X, j\notin X) \lor (i\in Y, j\notin Y)(i\in Z, j\notin Z)\right].$$

Design an algorithm based on SDP relaxation that solves this problem with approximation ration greater then $0.7$.

\soln

(We assume undirected graph $G$ just to write the notation $\set{i,j}$)
For the problem with two partitions, we had modeled the problem with having variables $x_{v}\in \set{\pm 1}$ for each vertex $v\in V$. For the corresponding problem with three partitions we will restrict each such variable to be a $2-$vector among $\pmb a_{1} \sett (1,0), \pmb a_{2} \sett\left(-\frac12,\frac{\sqrt3}{2}\right), \pmb a_{3} \sett \left(-\frac12,-\frac{\sqrt3}{2}\right)$. It is easy to verify that $\pmb a_{1}^{\top}\pmb a_{2} = \pmb a_{2}^{\top}\pmb a_{3} = \pmb a_{3}^{\top}\pmb a_{1} = -\frac12$. The three vertices $\pmb a_{1,2,3}$ stand for the three partitions $X,Y,Z$. Any edge $(u,v)\in E$ that gets assigned different classes of vertices, say $\pmb x_{u} = \pmb a_{1}, \pmb x_{v} = \pmb a_{2}$, contributes exactly $1 = \frac{2}{3}\left(1-\pmb a_{1}^{\top}\pmb a_{2}\right)$ to the cut value. If they are in the same class then $\pmb x_{u} = \pmb x_{v}$ and $\pmb x_{u}^{\top}\pmb x_{v} = 1$ giving a contribution of $0$ from the expression $\frac{2}{3}\left(1-\pmb x_{u}^{\top}\pmb x_{v}\right)$.

Let's make things formal now. Let $G=(V=[n],E)$ be the given graph. Introduce variables $\pmb x_{v}\in \R^{2}$, one for each $v\in V$, and constrain them $\pmb x_{v}\in \set{\pmb a_{1},\pmb a_{2}, \pmb a_{3}}$ where $\pmb a_{i}$ are as in the above paragraph. Given the above discussion, our problem is modeled as follows
\begin{equation}\label{max3cut}\begin{aligned}
f^{*}\sett \max_{\pmb x_{1},\cdots, \pmb x_{n}\in \R^{2}} &~~~~\frac{2}{3} \sum_{(i,j)\in E} (1-\pmb x_{i}^{\top}\pmb x_{j})\\
\text{s.t. } &~~~~ \pmb x_{i}\in \set{\pmb a_{1},\pmb a_{2}, \pmb a_{3}} \fa i\in V
\end{aligned}\end{equation}

We are essentially interested in $\ds\min_{\pmb x_{1},\cdots, \pmb x_{n}\in \R^{2}}\frac{2}{3} \sum_{(i,j)\in E} \pmb x_{i}^{\top}\pmb x_{j}$ s.t. $\pmb x_{i}\in \set{\pmb a_{1},\pmb a_{2}, \pmb a_{3}} \fa i\in V$.

To get an SDP relaxation, we relax our constraints to $\norm{\pmb x_{i}}{2} = 1\fa i\in V$ and $\pmb x_{i}^{\top}\pmb x_{j} \ge -\frac{1}{2}$. The last constraint gives the best angle separation among $3$ vectors on $\mathbb S^{2}$ in the following sense: if $t\in \R$ is such that $\pmb v_{1},\pmb v_{2},\pmb v_{3}\in \mathbb S^{2}$ satisfy $\pmb v_{i}^{\top}\pmb v_{j} \le t ~(\forall~ i\ne j)$ then $0\le \norm{\pmb v_{1}+\pmb v_{2}+\pmb v_{3}}{2}^{2} = 3 + 2\cdot 3 \cdot t \implies t\ge -1/2$. So we design an SDP with the rank$-2$ matrix $\begin{bmatrix}\pmb x_{1}^{\top} \\ \vdots \\ \pmb x_{n}^{\top}\end{bmatrix}_{n\times 2}\begin{bmatrix}\pmb x_{1} & \cdots & \pmb x_{n}\end{bmatrix}_{2 \times n} \succeq 0$ in mind:
\begin{equation}\label{sdprelax}\begin{aligned}
\frac{2m}{3}- f_{S} = \min_{X\in S^{n\times n}} &~~~~  \Tr\left[\frac{2}{3}QX\right]\\
\text{s.t. } &~~~~ X_{ii} = 1 \fa i\in V\\
&~~~~ X_{ij} \ge -\frac{1}{2} \fa i\ne j\in V\\
&~~~~ X \succeq 0
\end{aligned}\end{equation}
where $Q$ is a matrix whose $(i,j)^{\text{th}}$ entry is $1$ if $\set{i,j}\in E$ and $0$ otherwise, $S^{n\times n}$ denotes the space of al real symmetric $n\times n$ matrices, and $f_{S}$ is the optimal value obtained from SDP relaxation. 

Let's say the optimal solution of this SDP is attained at $X^{*}$, take a Cholesky factorization $X^{*} = V^{\top}V$ where $V\in \R^{r\times n}$ and $r=\rk V$. Let the columns of $V$ be $\pmb y_{1},\cdots, \pmb y_{n}\in \R^{r}$. In the rounding step, we choose random vectors $\pmb R_{1}, \pmb R_{2}, \pmb R_{3}\in \R^{r}$ such that each $\pmb R_{i,j}\sim \mathcal N(0,1)$ (for $1\le j \le r$) is chosen independently. These will give us the partitions, by rounding each $\pmb y_{i}$ to the component nearest among $\pmb R_{j}$. More precisely, we partition $V = V_{1}\sqcup V_{2}\sqcup V_{3}$ as follows:
\begin{align*}
V_{1} &\sett \set{i\in V \st \pmb y_{i}^{\top} \pmb R_{1} \ge \pmb y_{i}^{\top} \pmb R_{2},\pmb  y_{i}^{\top} \pmb R_{1} \ge \pmb y_{i}^{\top} \pmb R_{3}}\\
V_{2} &\sett \set{i\in V \st \pmb y_{i}^{\top} \pmb R_{2} \ge \pmb y_{i}^{\top} \pmb R_{1},\pmb  y_{i}^{\top} \pmb R_{2} \ge \pmb y_{i}^{\top}\pmb  R_{3}}\\
V_{3} &\sett \set{i\in V \st \pmb y_{i}^{\top} \pmb R_{3} \ge \pmb y_{i}^{\top} \pmb R_{2},\pmb y_{i}^{\top} \pmb R_{3} \ge \pmb y_{i}^{\top}\pmb  R_{1}}
\end{align*}
while breaking ties at random. In fact assign $\pmb x_{i} \sett \pmb a_{j}$ if $i\in V_{j}$.

Let $f_{R}$ denote the cut value produced by the above-mentioned randomized rounding. So $f_{R} = \sum_{\set{i,j}\in E} \pmb 1\left[\pmb x_{i}\ne \pmb x_{j}\right]$. We are interested in $\ds \E{f_{R}} = \sum_{\set{i,j}\in E} \P{\pmb x_{i}\ne \pmb x_{j}}$.










\newpage
\pb
The Ellipsoid algorithm we saw in the lecture solves convex programs assuming a separation oracle. Here, we want to show the opposite. To be more specific, consider the following two tasks regarding a convex body $\cK$:
\begin{itemize}
\item $\texttt{OPTIMIZE}(\cK):$ given a vector $c\in\R^{n}$, output $\arg\max\limits_{x\in\cK}c^{\top}x$;
\item $\texttt{SEPARATE}(\cK):$ given a point $x\in\R^{n}$, output either $x\in \cK$ or a separating hyperplane.
\end{itemize}

We are going to show that if for a specific convex body $\cK$, there is a polynomial time algorithm for $\texttt{OPTIMIZE}(\cK)$, then there is a polynomial time algorithm for $\texttt{SEPARATE}(\cK)$.
\begin{enumerate}[label = (\alph*)]
\item Suppose for a given $x$, we can solve the following LP with infinitely many constraints (finding the optimal $w$ and $T$). Show that we can use such an algorithm to solve $\texttt{SEPARATE}(\cK)$.
\begin{equation}\label{margin}
\begin{aligned}
\max_{w\in \R^{n},T\in\R} &~~ w^{\top} x-T\\
\text{s.t.} &~~ w^{\top} y \le T\fa y\in\cK\\
&~~ -1\le T\le 1
\end{aligned}
\end{equation}
\item Design a polynomial time separation oracle for the above LP using $\texttt{OPTIMIZE}(\cK)$, and conclude.
\end{enumerate}


\soln

\begin{enumerate}[label = (\alph*)]
\item Suppose the value of this LP is $>0$ and is attained at $(\overline w,\overline T)$. Then for any $y\in\cK$, $\overline w^{\top}y-\overline T \le 0$. This means that $x\notin \cK$.\\
Suppose $x\notin \cK$. Then there is a vector $w\in \R^{n}$ such that $w^{\top}x > 0$ and $w^{\top}y\le 0\fa y\in \cK$. This $(w,T=0)$ is feasible to \ref{margin} with objective $>0$. Thus its optimal value is $>0$.

Therefore $x\in \cK$ iff the optimal value of \ref{margin} is $\le 0$. If $\le 0$ with optimal $w=\overline w$, a separating hyperplane is $\overline w$ because of what is discussed above.

\item 
\end{enumerate}














\newpage
\pb
Describe separation oracles for the following convex sets. Your oracles should run in linear time, assuming that the given oracles run in linear time (so you can make a constant number of black-box calls to the given oracles).

\soln







\newpage
\pb






\soln



\newpage
\pb



\soln




\end{document}

