\documentclass[letterpaper,11pt]{article}

\usepackage{latexsym}
\usepackage[empty]{fullpage}
\usepackage{titlesec}
\usepackage{marvosym}
\usepackage[usenames,dvipsnames]{color}
\usepackage{verbatim}
\usepackage{enumitem}
\usepackage{hyperref}
\usepackage{fancyhdr}
\usepackage[english]{babel}
\usepackage{tabularx}
\usepackage{fontawesome5}
\usepackage{multicol}
\setlength{\multicolsep}{-3.0pt}
\setlength{\columnsep}{-1pt}
\input{glyphtounicode}
\usepackage{graphicx}
\usepackage{tikz}
\usepackage{libertinus}\usepackage[T1]{fontenc}
%----------FONT OPTIONS----------
% sans-serif
% \usepackage[sfdefault]{FiraSans}
% \usepackage[sfdefault]{roboto}
% \usepackage[sfdefault]{noto-sans}
% \usepackage[default]{sourcesanspro}

% serif
 %\usepackage{CormorantGaramond}
 %\usepackage{charter}

\def\mygap{-20pt}
\def\gap{-10pt}

\pagestyle{fancy}
\fancyhf{} % clear all header and footer fields
\fancyfoot{}
\renewcommand{\headrulewidth}{0pt}
\renewcommand{\footrulewidth}{0pt}

% Adjust margins
\addtolength{\oddsidemargin}{-0.4in}
\addtolength{\evensidemargin}{-0.4in}
\addtolength{\textwidth}{0.7in}
%\addtolength{\topmargin}{-0.5in}
%\addtolength{\textheight}{1in}
\setlength{\skip\footins}{8mm}

%\urlstyle{same}

\raggedbottom
\raggedright
\setlength{\tabcolsep}{0in}

% Sections formatting
\titleformat{\section}{
  \vspace{-4pt}\scshape\raggedright\Large\bfseries
}{}{0em}{}[\color{black}\titlerule \vspace{-5pt}]

% Ensure that generate pdf is machine readable/ATS parsable
\pdfgentounicode=1

%-------------------------
% Custom commands



\newcommand{\resumeItem}[1]{
  \item\small{
    {#1 \vspace{-2pt}}
  }
}

\newcommand{\classesList}[4]{
    \item\small{
        {#1 #2 #3 #4 \vspace{-2pt}}
  }
}

\newcommand{\resumeSubheading}[4]{
  \vspace{-1pt}
    \begin{tabularx}{\textwidth}[t]{X r}
      {\textbf{#1}} & \textbf{\small #2} \\
      {{\itshape\small #3} } & \textit{\small #4} \\
    \end{tabularx}\vspace{-2pt}
}



\newcommand{\resumeSubSubheading}[2]{
    \item
    \begin{tabular*}{0.97\textwidth}{l@{\extracolsep{\fill}}r}
      \textit{\small#1} & \textit{\small #2} \\
    \end{tabular*}\vspace{-7pt}
}

\newcommand{\resumeProjectHeading}[2]{
    \vspace{-6pt}\item
    \begin{tabular*}{1.001\textwidth}{l@{\extracolsep{\fill}}r}
      #1 & \textbf{\small #2}\\
    \end{tabular*}\vspace{-8pt}
}

\newcommand{\resumeSubItem}[1]{\resumeItem{#1}\vspace{-4pt}}

\renewcommand\labelitemi{$\vcenter{\hbox{\tiny$\bullet$}}$}
\renewcommand\labelitemii{$\vcenter{\hbox{\tiny$\bullet$}}$}

\newcommand{\resumeSubHeadingListStart}{\begin{itemize}[leftmargin=0.0in, label={}]}
\newcommand{\resumeSubHeadingListEnd}{\end{itemize}}
\newcommand{\resumeItemListStart}{\begin{itemize}[leftmargin=0.0in,label={}]}
\newcommand{\resumeItemListEnd}{\end{itemize}\vspace{-4pt}}
%\fontfamily{qcr}\selectfont 
%\usepackage[backend=bibtex]{biblatex}
\usepackage[
backend=biber,
style=alphabetic,%firstinits,
citestyle=ieee-alphabetic,
%natbib=true,
%uniquelist=false,
maxnames=10,
sorting=ynt
]{biblatex}
%\addbibresource{writeup/article/refs.bib}
%\title{\vspace{-1cm}}
\title{\textbf{Physics of Learning Theory}\\ Lecture $1$: Probability}
\usepackage{quiver}
\usepackage[nobottomtitles*]{titlesec}
\usepackage{titletoc}
%\titleformat{\section}[runin]
%  {\normalfont\Large\bfseries}
%  {}{0pt}{}%
%  [\ifthenelse{\equal{\thesection}{0}}{\\\vspace*{0pt}}{\space\thesection}]
%\author{{\Large NILAVA METYA} \\ 
%\href{mailto:nilava.metya@rutgers.edu}{nilava.metya@rutgers.edu}\\
%\href{mailto:nm8188@princeton.edu}{nm8188@princeton.edu}}
\author{Nilava Metya}
\date{\vspace{-0.7in}February $--$, $2025$}
%\newcommand{\pb}{\section{Problem}~\par}
%\newcommand{\soln}{\subsection*{Solution}}
\usepackage{pdfpages}
\usepackage{fancyhdr}
	\pagestyle{fancyplain}
	\fancyhf{}
	\fancyhead[R]{\thepage}
\newcommand{\fa}{~\forall~}
\begin{document}

\maketitle

\section{Introduction}
We will recall some probability theory and look at useful \textit{deviation} or \textit{concentration} bounds which are frequently used in analyzing algorithms (in learning theory). Recall that a (real-valued) random variable on a probability space $(\Omega,S,\mathbb P)$ is nothing but a `measurable function' $X:\Omega\to \R$. Here $\Omega$ is the universal or sample space where we think of events in, $S$ is a collection of events in $\Omega$ and $\mathbb P:S\to[0,1]$ assigns probability to each event in $S$. The space of events $S$ is constrained to satisfy some obvious rules like $\Omega$ is an event, if $A$ is an event then so is $\Omega\smallsetminus A$ and that a countable union of events is an event which makes it sensible to work with the concept of assigning probabilities to each event. We will often say that $\P{A}$ is the probability that event $A$ occurs. If $A=\set{a}$ is a singleton, we always write $\P{a}$ instead of $\P{\set{a}}$. The probability function $\mathbb P$ is also constrained to a couple of rules, namely, that the probability of the union of a mutually disjoint collection of events, which is an event, is the same as the sum of the probabilities of each of those events and that the probability that $\Omega$ occurs is $1$. Roughly a random variable is to be thought of as a way of assigning points of the sample space to real numbers which are \textit{really real} and are more tangible to work with, while respecting the rules of $S$. Such a random variable induces a map $X^{-1}:2^{\R}\to S$ by $X^{-1}(A) \sett \set{x\in \Omega\st X(x) \in S}$ for any $A\subseteq \R$, and hence induces a probability on $\R$ given by $\mathbb P_{\R}[A] = \P{X^{-1}(A)}$ where $A$ is any `measurable' subset of $\R$. The random variable being a `measurable function' precisely means that $X^{-1}(A)$ always lies in $S$.
\[\begin{tikzcd}
	S & {2^\R} \\
	{[0,1]}
	\arrow["{\mathbb P}"', from=1-1, to=2-1]
	\arrow["{X^{-1}}"', from=1-2, to=1-1]
	\arrow["{\mathbb P_\R}", dashed, from=1-2, to=2-1]
\end{tikzcd}\]

\subsection{Mean}
The average or mean of a random variable $X$, often denoted as $\E X$, $\mu(X)$, or simply $\mu$ when the context is clear,  is $\E X = \int_{\Omega}  X \d\mathbb P $. For the discrete case, which we will mostly be interested in, this boils down to $\E X = \sum\limits_{i\in \Omega}X(i)\P{i}$. Note that if $X$ is an indicator random variable for event $A$, that is, $X=1$ if $A$ occurs and $0$ otherwise, then $\E X = \P A$.

\begin{ex}
Consider tossing a fair coin. Here $\Omega=\set{\text H, \text T}$. The probability function is $\P{\varnothing}=0, \P{\text H}=\P{\text T}=0.5, \P{\set{\text H,\text T}}=1$. A natural random variable to consider is $X(i) = \pmb 1_{\text H} \sett \begin{cases}1&\text{ if } i=\text H\\0&\text{ if } i=\text T\end{cases}$. The corresponding probability induced on $\R$ is given by $\mathbb P_{\R}[A] = \begin{cases}
0 &\text{ if } 0\notin A,1\notin A\\
0.5 &\text{ if } 0\in A,1\notin A\\
0.5 &\text{ if } 0\notin A,1\in A\\
1 &\text{ if } 0\in A,1\in A\\
\end{cases}$. In this case, $\E X = 1\cdot \P{\text H} + 0\cdot \P{\text T} = 0.5$
\end{ex}

\begin{ex}
Consider tossing $n$ fair coins sequentially and independently. Here $\Omega=\set{\text H, \text T}^{n}$. So the singleton outcomes are tuples of $\text H,\text T$. The probability function is given by $\P{\pmb x}=2^{-n}$ for any element $x\in {\Omega}$ and then extending by countable additivity of $\mathbb P$. Consider $n$ random variables $X_{1},\cdots,X_{n}$ where $X_{i}(\pmb x) \sett \begin{cases}1&\text{ if } x_{i}= \text H\\0&\text{ if } x_{i}=\text T\end{cases}$. Each $X_{i}$ is the same random variable as the previous example after looking at the $i^{\text{th}}$ coordinate. A natural variable to consider is the total number of heads obtained in one round of tossing, that is $X = X_{1}+\cdots+X_{n}$. The corresponding probability induced on $\R$ is given by $\mathbb P_{\R}[k] = \begin{cases}
{n\choose k}2^{-n} &\text{ if } k\in \set{0,\cdots,n}\\
0 &\text{ otherwise}
\end{cases}$ and extend by countable additivity. Here $\E X = \frac{n}{2}$.
\end{ex}

One useful result used for calculating expectations of sums of random variables is that if $a,b\in \R$ and $X,Y$ are random variables then $\E{aX+bY}=a\E{X}+b\E{Y}$. It's worthy to note that sums and scalings of random variables are random variables. This result does \textbf{not} depend on `independence' of $X,Y$. Independence plays an important role for the average of products of random variables (which is a random variable). We say random variables $X_{1},\cdots,X_{n}$ are (\textit{mutually}) \textit{independent} if $\P{\bigcap_{i=1}^{n}\set{X_{i}\le a_{i}}} = \prod_{i=1}^{n}\P{X_{i}\le a_{i}}\fa a_{i}\in\R$. This is a stronger notion than \textit{pairwise} independence where we demand that only every pair of them are independent. Note that mutual independence implies pairwise independence. If $X,Y$ are independent then $\E{XY} = \E{X}\E{Y}$. 

For a random variable $X\ge 0$ and $a\in\R$ let $Y$ be the indicator random variable indicating whether $X\ge a$, that is, $Y$ is $1$ if $X\ge a$ and $0$ otherwise. Then clearly $X \ge aY$. Indeed if $X\ge a$ then $Y=1$ so $X\ge aY$ and if $X<a$ then $Y=0$ so that $X \ge 0 = aY$. Expectation preserves inequalities, so $\E{X} \ge a\E{Y} = a\P{X\ge a}$. This establishes
\begin{thm}[Markov's inequality]\label{thm:markov}
If $X$ is a non-negative random variable and $a\in\R$ then $\P{X\ge a}\le \dfrac{\E X}{a}$.
\end{thm}

\subsection{Variance}
Let's come to deviation now. One natural way to measure \textit{deviation} is to look how on average much a random variable deviates either way from its mean (behavior). To look for deviation in either direction of $\E{X}$ we consider the random variable $(X-\E X)^{2}$. Define the variance of a random variable $X$ as $\Var X \sett \E{(X-\E X)^{2}}$. One useful result to compute variance is that if $X,Y$ are independent then $\Var{aX+bY} = a^{2}\Var{X} + b^{2}\Var{Y}$. This extends to $n$ pairwise independent random variables. Another useful result is $\Var{X} = \E{X^{2}} - \E{X}^{2}$.

Applying \Cref{thm:markov} to $(X-\E X)^{2}\ge 0$ gives 
\begin{thm}[Chebyshev's inequality]\label{thm:chebyshev}
If $X$ is a random variable and $a\in\R_{{\ge 0}}$ then $\P{\abs{X-\E{X}}\ge a}\le \dfrac{\Var X}{a^{2}}$.
\end{thm}

\subsection{Higher moments}
One might just ask why stop at the second power to measure deviation. What about the random variable $(X-\E X)^{k}$ for $k\ge 2$? These are called higher centeral moments. Note that $\E{(X-\E X)^{k}} = 0$ when $k$ is odd and the distribution of $X$ is symmetric about $\E{X}$. So it makes sense to consider the random variables $\mu_{k} \sett \abs{X-\E X}^{k}$ instead. If we have access to such numbers, we can use the same trick as the proof of Chebyshev's inequality and get $\P{\abs{X-\E X}\ge a} \le \dfrac{\mu_{k}}{a^{k}}$. Knowing all higher moments means that we know something known as the `characteristic function' (not yet defined) of $X$ which uniquely determines $X$. But our aim was the study deviations using small information. Generally, higher moments are not known.



\end{document}

