\documentclass[12pt]{amsart}
\usepackage[letterpaper, portrait, left = 1in, right = 1in, top = 1.2in, bottom=1.5in]{geometry} 
%\usepackage{setspace} \doublespacing
%\usepackage[letterpaper, portrait, margin=1.3in]{geometry}
\usepackage[table,xcdraw]{xcolor}
\usepackage{amssymb}
\usepackage{amsfonts}
\usepackage{longtable}
\usepackage{amsmath,amsthm}
\usepackage{enumitem}
\usepackage[utf8]{inputenc}
\usepackage{mathtools}
\usepackage{graphicx}
\usepackage{parskip}
\usepackage{multicol}
\usepackage{listings}
\usepackage[skip=0.25pt]{caption}
\usepackage[mathscr]{euscript}
\usepackage{quiver}
\setlength{\parindent}{0pt}
\usepackage{thm-restate}
\definecolor{vividburgundy}{rgb}{0.62, 0.11, 0.21}
\usepackage[driverfallback=hypertex,pagebackref=false,colorlinks,citecolor=vividburgundy]{hyperref}
\usepackage[capitalize]{cleveref}
%\usepackage[cmintegrals,cmbraces]{newtxmath}
%\usepackage{ebgaramond-maths}

%\usepackage{fourier}
%----------FONT OPTIONS----------
% sans-serif
%\usepackage[sfdefault]{FiraSans}
 %\usepackage[sfdefault]{roboto}
% \usepackage[sfdefault]{noto-sans}
%\usepackage[default]{sourcesanspro}

% serif
%\usepackage{CormorantGaramond}

%\usepackage{charter}
\usepackage[T1]{fontenc}
\usepackage{cleveref}
\definecolor{dg}{RGB}{10, 100, 10}
\setlength{\parindent}{0in}
\renewcommand{\qed}{$\hfill\blacksquare$}
% \newtheoremstyle{style}{2pt}{1pt}{\normalfont}{}{\bfseries}{\\}{0cm}{}
% \theoremstyle{style}
\newtheorem{lemma}{Lemma}[section]
% \newtheorem{lemma}{Lemma}
\newtheorem{thm}[lemma]{Theorem}
\newtheorem{prop}[lemma]{Proposition}
\newtheorem{cor}[lemma]{Corollary}
\newtheorem{conj}[lemma]{Conjecture}
\newtheorem{cl}[lemma]{Claim}
\newtheorem{rmk}{Remark}
\newtheorem{defn}[lemma]{Definition}
\newtheorem{qs}{Question}
\newtheoremstyle{styleS}{}{}{\color{dg}}{}{\color{dg}\bfseries}{. }{0cm}{}
\theoremstyle{styleS}
\newtheorem*{sol}{Solution}
\newtheoremstyle{style1}{}{}{\normalfont}{}{\bfseries}{. }{0cm}{}
\theoremstyle{style1}
\newtheorem{prob}{Problem}[section]
\newtheorem*{prb}{Problem}
\newtheoremstyle{style2}{1pt}{4pt}{\normalfont}{}{\itshape}{. }{0cm}{}
\theoremstyle{style2}
\newtheorem{ex}[lemma]{Example}
\newtheorem*{pf}{Proof}

\newcommand{\norm}[2]{
\left\lVert #1 \right\rVert_{#2}
}
\usepackage{mathrsfs}
%\usepackage[table,xcdraw]{xcolor}
\usepackage{booktabs}
\usepackage{tikz}
\usetikzlibrary{matrix}
\renewcommand{\l}{\ell}


\newcommand{\fA}{{\mathfrak{A}}}   \newcommand{\fB}{{\mathfrak{B}}}
\newcommand{\fC}{{\mathfrak{C}}}   \newcommand{\fD}{{\mathfrak{D}}}
\newcommand{\fE}{{\mathfrak{E}}}   \newcommand{\fF}{{\mathfrak{F}}}
\newcommand{\fG}{{\mathfrak{G}}}   \newcommand{\fH}{{\mathfrak{H}}}
\newcommand{\fI}{{\mathfrak{I}}}   \newcommand{\fJ}{{\mathfrak{J}}}
\newcommand{\fK}{{\mathfrak{K}}}   \newcommand{\fL}{{\mathfrak{L}}}
\newcommand{\fM}{{\mathfrak{M}}}   \newcommand{\fN}{{\mathfrak{N}}}
\newcommand{\fO}{{\mathfrak{O}}}   \newcommand{\fP}{{\mathfrak{P}}}
\newcommand{\fQ}{{\mathfrak{Q}}}   \newcommand{\fR}{{\mathfrak{R}}}
\newcommand{\fS}{{\mathfrak{S}}}   \newcommand{\fT}{{\mathfrak{T}}}
\newcommand{\fU}{{\mathfrak{U}}}   \newcommand{\fV}{{\mathfrak{V}}}
\newcommand{\fW}{{\mathfrak{W}}}   \newcommand{\fX}{{\mathfrak{X}}}
\newcommand{\fY}{{\mathfrak{Y}}}   \newcommand{\fZ}{{\mathfrak{Z}}}

\newcommand{\cA}{{\mathcal{A}}}   \newcommand{\cB}{{\mathcal{B}}}
\newcommand{\cC}{{\mathcal{C}}}   \newcommand{\cD}{{\mathcal{D}}}
\newcommand{\cE}{{\mathcal{E}}}   \newcommand{\cF}{{\mathcal{F}}}
\newcommand{\cG}{{\mathcal{G}}}   \newcommand{\cH}{{\mathcal{H}}}
\newcommand{\cI}{{\mathcal{I}}}   \newcommand{\cJ}{{\mathcal{J}}}
\newcommand{\cK}{{\mathcal{K}}}   \newcommand{\cL}{{\mathcal{L}}}
\newcommand{\cM}{{\mathcal{M}}}   \newcommand{\cN}{{\mathcal{N}}}
\newcommand{\cO}{{\mathcal{O}}}   \newcommand{\cP}{{\mathcal{P}}}
\newcommand{\cQ}{{\mathcal{Q}}}   \newcommand{\cR}{{\mathcal{R}}}
\newcommand{\cS}{{\mathcal{S}}}   \newcommand{\cT}{{\mathcal{T}}}
\newcommand{\cU}{{\mathcal{U}}}   \newcommand{\cV}{{\mathcal{V}}}
\newcommand{\cW}{{\mathcal{W}}}   \newcommand{\cX}{{\mathcal{X}}}
\newcommand{\cY}{{\mathcal{Y}}}   \newcommand{\cZ}{{\mathcal{Z}}}

\newcommand{\sA}{{\mathscr{A}}}   \newcommand{\sB}{{\mathscr{B}}}
\newcommand{\sC}{{\mathscr{C}}}   \newcommand{\sD}{{\mathscr{D}}}
\newcommand{\sE}{{\mathscr{E}}}   \newcommand{\sF}{{\mathscr{F}}}
\newcommand{\sG}{{\mathscr{G}}}   \newcommand{\sH}{{\mathscr{H}}}
\newcommand{\sI}{{\mathscr{I}}}   \newcommand{\sJ}{{\mathscr{J}}}
\newcommand{\sK}{{\mathscr{K}}}   \newcommand{\sL}{{\mathscr{L}}}
\newcommand{\sM}{{\mathscr{M}}}   \newcommand{\sN}{{\mathscr{N}}}
\newcommand{\sO}{{\mathscr{O}}}   \newcommand{\sP}{{\mathscr{P}}}
\newcommand{\sQ}{{\mathscr{Q}}}   \newcommand{\sR}{{\mathscr{R}}}
\newcommand{\sS}{{\mathscr{S}}}   \newcommand{\sT}{{\mathscr{T}}}
\newcommand{\sU}{{\mathscr{U}}}   \newcommand{\sV}{{\mathscr{V}}}
\newcommand{\sW}{{\mathscr{W}}}   \newcommand{\sX}{{\mathscr{X}}}
\newcommand{\sY}{{\mathscr{Y}}}   \newcommand{\sZ}{{\mathscr{Z}}}

\newcommand{\ta}{{\tilde{a}}}   \newcommand{\tb}{{\tilde{b}}}
\newcommand{\tc}{{\tilde{c}}}   \newcommand{\td}{{\tilde{d}}}
\newcommand{\te}{{\tilde{e}}}   \newcommand{\tf}{{\tilde{f}}}
\newcommand{\tg}{{\tilde{g}}}   
\newcommand{\ti}{{\tilde{i}}}   \newcommand{\tj}{{\tilde{j}}}
\newcommand{\tk}{{\tilde{k}}}   \newcommand{\tl}{{\tilde{l}}}
\newcommand{\tm}{{\tilde{m}}}   \newcommand{\tn}{{\tilde{n}}}
		         	\newcommand{\tp}{{\tilde{p}}}
\newcommand{\tq}{{\tilde{q}}}   \newcommand{\tr}{{\tilde{r}}}
\newcommand{\ts}{{\tilde{s}}}   
\newcommand{\tu}{{\tilde{u}}}   \newcommand{\tv}{{\tilde{v}}}
\newcommand{\tw}{{\tilde{w}}}   \newcommand{\tx}{{\tilde{x}}}
\newcommand{\ty}{{\tilde{y}}}   \newcommand{\tz}{{\tilde{z}}}

\newcommand{\red}{{\color{red}red}}
\newcommand{\blue}{{\color{blue}blue}}

\newcommand{\into}{\hookrightarrow}
\newcommand{\onto}{\twoheadrightarrow}
\newcommand\N{\ensuremath{\mathbb{N}}}

%\newcommand\L{\ensuremath{\mathbb{L}}}
\newcommand{\bP}{\mathbb{P}}
\newcommand\M{\ensuremath{\mathbb{M}}}
\newcommand\R{\ensuremath{\mathbb{R}}}
\newcommand\Z{\ensuremath{\mathbb{Z}}}
\renewcommand\O{\ensuremath{\emptyset}}
\newcommand\Q{\ensuremath{\mathbb{Q}}}
\newcommand\C{\ensuremath{\mathbb{C}}}
\newcommand{\K}{\ensuremath{\mathbb{K}}}
\newcommand\F{\ensuremath{\mathbb{F}}}
\newcommand{\aff}{\ensuremath{\mathbb{A}}}
\newcommand{\proj}{\ensuremath{\mathbb{P}}}
\newcommand{\dd}{\mathrm{d}}
\newcommand{\m}{\ensuremath{\mathfrak{m}}}
\newcommand{\p}{\ensuremath{\mathfrak{p}}}
\newcommand{\n}{\ensuremath{\mathfrak{n}}}
\renewcommand{\phi}{\varphi}
\renewcommand{\qedsymbol}{\ensuremath{\blacksquare}}
%\newcommand{\st}{\;|\;}
\newcommand{\st}{%
  \nonscript\;
  \ifnum\currentgrouptype=16
    \;\middle|\;
  \else
    \;|\;
  \fi
  \nonscript\;}
\newcommand{\ltr}{\par \noindent \framebox[1\width]{ $\implies$ } \hspace{.2cm}}
\newcommand{\rtl}{\par \noindent \framebox[1\width]{ $\impliedby$ } \hspace{.2cm} }
\newcommand{\abs}[1]{\left| #1 \right|}
\newcommand{\inner}[2]{\left\langle #1, #2 \right\rangle}
\newcommand{\E}[1]{\mathbb E\left[ #1 \right]}
\newcommand{\e}[1]{\exp\left( #1 \right)}
\renewcommand{\P}[1]{\mathbb P\left[ #1 \right]}
\newcommand{\Var}[1]{\text{Var}\left[ #1 \right]}
\newcommand*\circled[1]{\tikz[baseline=(char.base)]{
            \node[shape=circle,draw,inner sep=2pt] (char) {#1};}}
\newcommand{\ds}{\displaystyle}

\DeclareMathOperator{\sym}{Sym}
\DeclareMathOperator{\mds}{MDS}
\DeclareMathOperator{\Tor}{Tor}
\DeclareMathOperator{\Ext}{Ext}
\DeclareMathOperator{\adj}{adj}
\DeclareMathOperator{\Tr}{Tr}
\DeclareMathOperator{\GL}{GL}
%\DeclareMathOperator{\Tr}{Tr}
\DeclareMathOperator{\orbit}{Or}
\DeclareMathOperator{\stab}{Stab}
\DeclareMathOperator{\fix}{Fix}
\DeclareMathOperator{\re}{Re}
\DeclareMathOperator{\im}{Im}
\DeclareMathOperator{\ord}{Ord}
\DeclareMathOperator{\mspec}{mSpec}
\DeclareMathOperator{\spec}{Spec}
\DeclareMathOperator{\frob}{Frob}
\DeclareMathOperator{\id}{Id}
\DeclareMathOperator{\colim}{colim}
\DeclareMathOperator{\loc}{loc}
\DeclareMathOperator{\res}{Res}
\DeclareMathOperator{\rad}{rad}
\DeclareMathOperator{\Res}{Res}
\DeclareMathOperator{\diam}{diam}
\DeclareMathOperator{\arcsec}{arcsec}
\DeclareMathOperator{\arccot}{arccot}
\DeclareMathOperator{\len}{len}
\DeclareMathOperator{\area}{area}
\DeclareMathOperator{\vol}{vol}
\DeclareMathOperator{\ev}{ev}
\DeclareMathOperator{\sgn}{sgn}
\DeclareMathOperator{\supp}{supp}
\DeclareMathOperator{\diff}{d}
\DeclareMathOperator{\Dom}{Dom}
\DeclareMathOperator{\rk}{rank}
\renewcommand{\d}{\diff}
\let\oldend\endlinechar
\renewcommand{\endlinechar}{\oldend}
\newcommand{\open}{\underset{\text{open}}{\subset}}
\newcommand{\divides}{\mathbin{|}}
\newcommand{\set}[1]{\ensuremath{\left\{#1\right\}}}
\newcommand{\sett}{\coloneqq}
\newcommand*\isomap{%
  \xrightarrow{\raisebox{-0.9ex}[0ex][0ex]{$\sim$}}%
}
\renewcommand{\epsilon}{\varepsilon}
\newcommand{\fa}{~\forall~}
%\usepackage[nobottomtitles*]{titlesec}
\usepackage{titletoc}
%\titleformat{\section}[runin]
%{\normalfont\Large\bfseries}
%{}{0pt}{}%
%[\ifthenelse{\equal{\thesection}{0}}{\\\vspace*{0pt}}{\space\thesection}]
\newcommand{\sint}{\sin\theta}
\newcommand{\cost}{\cos\theta}
\newcommand{\tant}{\tan\theta}
\newcommand{\lb}{\left[}
\newcommand{\rb}{\right]}
\newcommand{\lp}{\left(}
\newcommand{\rp}{\right)}
\newcommand{\br}[1]{\lb#1\rb}
\newcommand{\pa}[1]{\lp#1\rp}
\usepackage{pdfpages}
\usepackage{fancyhdr}
	\pagestyle{fancyplain}
	\fancyhf{}
	\fancyhead[C]{\thepage}

\newcommand{\eto}{\stackrel{=}{\to}}
\fontfamily{qcr}\selectfont 
%\usepackage[backend=bibtex]{biblatex}
\usepackage[
backend=biber,
style=alphabetic,%firstinits,
citestyle=ieee-alphabetic,
%natbib=true,
%uniquelist=false,
maxnames=10,
sorting=ynt
]{biblatex}
\addbibresource{writeup/article/refs.bib}
%\title{\vspace{-1cm}}
\title{The Probabilistic Method}
\usepackage{quiver}
\author{Instructor: Noga Alon\\Notes by Nilava Metya}
\date{Spring 2024}
\makeatletter
\renewcommand{\@chapapp}{Lecture}
\makeatother

\begin{document}

\maketitle
\tableofcontents

\chapter{01/30/2024}

\section{Philosophy}
Main philosophy of the probabilistic method: To prove existence of a structure (or a substructure of a given one), define a probability space of structures, and show that a random point in it satisfies the required properties with positive (often high) probability.

We will look at two examples today.

\section{Example: Ramsey Theory}

\begin{defn}[Ramsey numbers]
For $k,\l\geq 1$, let $r=r(k,\l)$ be the smallest integer, if there exists any, satisfying the following property: for every coloring of edges of $G=K_r$ (the complete graph on $r$ nodes) by \red~and \blue, either $\exists$ a  blue $K_k\subseteq G$ or a red $K_{\l}\subseteq G$.
\end{defn}
\begin{ex}
$r(3,3)=6$.
\end{ex}

A special case of Ramsey's theorem says that $\exists r(k,l)<\infty\forall k,l$. The proof, by induction (using Erd\"os-Szekeres theorem), gives $\ds r(k,\l)\leq\binom{k+\l-2}{k-1}$. In particular, $r(k,k) \leq \binom{2k-2}{k-1} <4^k$.

\begin{rmk}
The following are easy to observe: $r(k,\l)=r(l,k), r(1,\l)=1, r(2,\l)=\l$.
\end{rmk}

All the exactly known Ramsey numbers for $\l\geq k\geq 3$ are
$r(3,3)=6, r(3,4)=9, r(3,5)=14, r(3,6)=18, r(3,7)=23, r(3,8)=28, r(3,9)=36, r(4,4)=18, r(4,5)=25$. It is only known that $41\leq r(3,10)\leq 42, 36\leq r(4,6)\leq 40, 43\leq r(5,5)\leq 48$, and some similar bounds for other Ramsey numbers.

\begin{thm}[Erdos '47]
If $\ds\binom n k 2^{1-\binom k 2} < 1$ then $r(k,k)>n$. Therefore $r(k,k)\geq \left[1-o(1)\right]\frac{k}{e} 2^{\frac{k-1}2}$.
\end{thm}
\begin{pf}
Take the complete graph on $n$ labelled vertices $[n] = \set{1,\cdots,n}$. Color each edge $\set{i,j}$ (for $1\leq i<j\leq n$) randomly uniformly and independently either \red~or \blue. For fixed $K\subseteq[n]$ with $k=\abs K$, the probability that the graph induced by $K$ is monochromatic is $\ds{\color{red}2^{-\binom k 2}} + {\color{blue}2^{-\binom k 2}} = 2^{1-\binom k 2}$. So 
\begin{align*}
\ds\P{\exists \text{ such monochromatic }K} &\leq \sum_{\substack{K\subseteq [n]\\ \abs K = k}} \P{K \text{ induces a monochromatic graph}} \\
&= \binom n k 2^{1-\binom k 2} \stackrel{\text{given}}{<} 1.
\end{align*}
Therefore, $\ds\P{\not\exists \text{ such monochromatic }K} >0$. This means $r(k,k)>n$, which proves the first part.

Now, \begin{align*}
\binom n k 2^{1-\binom k 2} \leq 2 \left(\frac{en}{k}\right)^k\cdot 2^{-\binom k2} = 2\left(\frac{en}{2^{ \frac{k-1}{2}}\cdot k }\right)^k
\end{align*}
where the first inequality is due to $\ds\binom a b \leq \left(\frac{ea}{b}\right)^b$. If $\frac{en}{2^{ \frac{k-1}{2}}\cdot k }<1-\epsilon$ then for $k>k_0(\epsilon)$ for some $k_0(\epsilon)$, the RHS is $<1$. This implies that $r(k,k) \geq \left[1-o(1)\right]\frac{k}{e} 2^{\frac{k-1}2}$.\footnote{
Explanation for the last `implies': We note that for every $n$ satisfying the given condition, we have $r(k,k)>n$. Now for any $n<\left[1-\epsilon\right]\frac{k}{e} 2^{\frac{k-1}2}$, the condition is satisfied. Thus, $r(k,k)$ is more than all such $n$'s, which is written as $\left[1-o(1)\right]\frac{k}{e} 2^{\frac{k-1}2}$.
}
\qed\end{pf}

\rmk{
The lower bound was improved only by a factor of two since $1947$.\\
The upper bound was improved several times, last time in $2023$ by Campos, Griffiths, Morris, Sahasrabudhe to $(4-\epsilon)^k$, for an absolute constant $\epsilon>0$.\\
Open: Does $\lim r(k,k)^{1/k}$ exist (for USD 100)? If exists, find it (for USD 250).
}

\rmk{
Open problem: Find an explicit coloring showing $r(k,k)>1.0001^k$.
}

\rmk{This proof provides a randomized algorithm for finding a coloring that shows  $r(k,k) > \left\lfloor\sqrt{2^k}\right\rfloor$. But given such a coloring, we don't know how to efficiently check that $\not\exists$ a monochromatic $K_k$.
}

\section{Example: Dominating Sets}

\begin{defn}
If $G=(V,E)$ is a graph, we say $S\subseteq V$ is dominating if $\forall v\in V\smallsetminus S\exists u\in S$ such that $ \set{u,v}\in E$.
\end{defn}

\ex{
The set of bold vertices in \begin{tikzcd}
	\circ & \bullet & \circ \\
	\bullet & \circ
	\arrow[no head, from=1-1, to=1-2]
	\arrow[no head, from=1-2, to=1-3]
	\arrow[no head, from=1-2, to=2-2]
	\arrow[no head, from=1-1, to=2-1]
	\arrow[no head, from=2-1, to=2-2]
	\arrow[no head, from=1-1, to=2-2]
\end{tikzcd} form a dominating set.
}

\begin{thm}
If $G=(V,E)$ is a graph with $\abs V = n$ and minimum degree $\delta$, then it has a dominating set of size at most $\ds n\cdot \frac{1+\ln(1+\delta)}{1+\delta}$.
\end{thm}

\begin{pf}
Let $p=\frac{\ln(1+\delta)}{1+\delta}$. Clearly $p\in [0,1]$. Let $X\subseteq V$
 be a random subset of $V$ obtained by choosing each $v\in V$ to randomly and independently lie in $X$ with probability $p$. Since $X$ is not necessarily a dominating set, we can \textit{alter} it by $$Y_X\sett\set{v\in V\smallsetminus X\st \not\exists u\in X \text{ with } \set{u,v}\in E}.$$
By construction, $X\sqcup Y_X$ is a dominating set (note that they are disjoint).

Let's estimate the expected size of $X\cup Y_X$. First observe that $\E{\abs{X\cup Y_X}} = \E{\abs X + \abs {Y_X}}$ due to disjointness, and this is further equal to $\E{\abs X} + \E{\abs {Y_X}}$ by linearity of expectation.\\
$\abs X$ is a sum of independent indicators, one for each vertex which takes $1$ with probability $p$ and $0$ with probability $1-p$. So $\E{\abs X} = np$.\\
Note that $\P{v\in Y_X} = \P{v\notin X}\cdot \P{\text{no neighbor of } v \text{ is in } X} = (1-p)^{d_v} \leq (1-p)^{1+\delta} =\frac{1}{1+\delta}$ where $d_v$ is the degree of $v$ in $G$. Again $\abs{Y_X} = \sum_{v\in V} \pmb 1_{v\in Y_X}$ whence $\E{\abs {Y_X}} \leq \frac n {1+\delta}$. \\
This means $\E{\abs{X\cup Y_X}} \leq n\left[\frac{1+\ln(1+\delta)}{1+\delta}\right]$. Since the `average size' of a dominating set is less than or equal to the given quantity, $\exists$ a choice of $X$ such that $X\cup Y_X$ is a dominating set of size at most $\ds n\cdot \frac{1+\ln(1+\delta)}{1+\delta}$.
\qed\end{pf}

\rmk{We used \textit{linearity of expectation}: $\E{X+Y} = \E X + \E Y$. We also used \textit{alteration}: making a change after initial random choice, in this case we added $Y_X$ to $X$. (To be discussed more)}

\rmk{Here $\exists$ an efficient algorith to find such a dominating set. Start with $\varnothing$ and keep adding vertices that dominate maximum of yet non-dominated vertices.}

\rmk{Estimate is essentially that for $n\gg\delta\gg1$.}



\chapter{02/01/2024}

Examples continued from last lecture.

\section{Example: Hypergraph $2-$coloring}

\begin{defn}
A \textit{hypergraph} is a pair $H=(V,E)$ of (finitely many) vertices $V$ and edges $E\subseteq 2^V$.

We say a hypergraph is \textit{$n-$uniform} if $\abs e=n\forall e\in E$. In particular, graphs are $2-$uniform hypergraphs.

We say a hypergraph is said to be \textit{$2-$colorable} if there exists a coloring of $V$ with \red~and \blue~with no monochromatic edge.
\end{defn}

We define the quantity $$m(n)\sett \min \set{\abs E \st (V,E) \text{ is } n- \text{uniform hypergraph and not } 2-\text{colorable}}$$ and interested in its asymptotics.

It is known that $m(1)=1, m(2)=3, m(3)=17, m(4)=23$ and for $n\geq 5$, $m(n)$ are unknown.

\begin{prop}
$m(n)\geq 2^{n-1}$ for $n\geq 2$.
\end{prop}

\pf{
For the sake of contradiction, let $H=(V,E)$ be $n-$uniform with $\abs E < 2^{n-1}$. We will show that $H$ is $2-$colorable. Color randomly each vertex independently either red or blue with probability half for each color. For each edge $e\in E$, let $A_e$ be the event that $e$ is monochromatic. Then $\P{A_e} = 2\cdot \left(\frac{1}{2}\right)^n = 2^{1-n}$. This means that $\P{\cup_{e\in E}A_e} \leq \sum_{e\in E}\P{A_e} = \abs E \cdot 2^{1-n}$ which is less than $1$ by assumption. This means that the event that no edge is monochromatic has positive probability, implying that there is a coloring for which there is no monochromatic edge. By definition, this is a $2-$coloring. 
\qed}

\rmk{The proof for lower bound of $r(k,k)$ is a special case. Take $n={k\choose 2}$. Vertices of the hypergraph are $E(K_n)$ and hyperedges are collections of $k\choose 2$ edges of $K_n$ that form a $k-$clique. So number of hyperedges is $n\choose k$.
}

\rmk{It can be shown that $m(n)\leq \cO(n^22^{n-1})$, that is $\exists c>0$ such that $m(n)\leq cn^22^{n-1}$ for all large $n$. Indeed if we take $2n^2$ vertices and $cn^22^{n-1}$ random subsets of size $n$, then with positive probability, every set of $n^2$ vertices contains an edge. So not $2-$colorable.\\
Note that the interesting quantity here is $\frac{m(n)}{2^{n-1}}$ which is the expected number of monochromatic edges in a random coloring. Thus $\ds1\leq \frac{m(n)}{2^{n-1}}\leq \cO(n^2)$.
}


Lower bound for $\frac{m(n)}{2^{n-1}}$ has been improved by Beck, by Radhakrishnan + Srinivasan. Best (short) proof is by Cherkashin and Kozik which is the following.

\begin{thm}
If $\exists k\geq 1, 0\leq p\leq 1$ such that $k(1-p)^n+k^2p<1$ then $m(n)>k\cdot 2^{n-1}$.
\end{thm}

\pf{
Let $n,k,p$ be as in the hypothesis of the theorem we're proving. Let $H=(V,E)$ be an $n-$uniform graph with $\abs E=k\cdot 2^{n-1}$. For each $v\in V$ pick $x_v\in [0,1]$ uniformly randomly. (We can assume that these $x_v$'s are unique because any two of them are equal with $0$ probability). These $x_v$'s define an ordering on the vertices, that is, we say $v<u$ iff $x_v<x_u$.

Now go over the vertices in increasing order and color each vertex \blue~ unless forced to color it \red~(namely, the vertex appears as the last vertex in an otherwise blue edge). By construction, there is no blue edge. But there can be a red edge. Let's look at probability that such a thing happens.

Define $L=\left[0,\frac{1-p}{2}\right), M=\left[\frac{1-p}2,\frac{1+p}2\right), R=\left[\frac{1+p}2,1\right]$. Let $A_e$ be the event that edge $e\in E$ is fully contained in $L$ or fully contained in $R$, and define $A\sett \bigcup_{e\in E}A_e$. Then $\P{A_e} = \P{x_v\in L\forall v\in e} + \P{x_v\in R\forall v\in e} = 2\P{x_v\in L\forall v\in e} = 2 \cdot\left(\frac{1-p}{2}\right)^n$. Thus \begin{align*}
\P A &\leq \sum_{e\in E} \P{A_e} \\
&\leq k\cdot 2^{n-1} \cdot 2 \cdot \left(\frac{1-p}{2}\right)^n \\
&= k(1-p)^n.
\end{align*}
Suppose the event $\ds\bigcup_{e\in E} A_e$ does not happen and there is a red edge. The former means every edge has one vertex each in at least two of $L,M,R$. Consider the first red edge $\color{red} e_0$, that is, the edge $\color{red} e$ with lowest value of $\ds\min_{v\in {\color{red}e}}x_v$ among red edges. Let $\color{red}v_0$ be the first vertex in $\color{red} e_0$. Clearly ${\color{red} v_0}\notin R$, else $\color{red} e_0$ would be completely in $R$. Also, ${\color{red} v_0}\notin L$ because otherwise $\color{red} v_0$ is the last edge of some otherwise blue edge which would hence completely be in $L$. Thus ${\color{red}v_0}\in M$. Say $\color{red} v_0$ is the last vertex of ${\color{blue} f_0}\in E$. Altogether, we care that there are two edges ${\color{red}e_0},{\color{blue}f_0}$ with $e_0\cap f_0 = \set{v_0}$ and $v_0\in M$, also called a \textit{conflicting} pair of edges. Also in this case, the probability that $v_0$ is the last vertex of ${\color{blue}f_0}$ is $\P{x_u \leq x_{v_0} \forall u\in f_0\smallsetminus\set{v_0}} = x_{v_0}^{n-1}$, and the probability that $v_0$ is the first vertex of ${\color{red}e_0}$ is $\P{x_u \geq x_{v_0} \forall u\in e_0\smallsetminus\set{v_0}} = (1-x_{v_0})^{n-1}$, because $\abs{e_0} = \abs{f_0} = n$ (by $n-$regularity of $H$). Thus 
\begin{align*}
\P{A^c\cap \set{\exists\text{ red edge}}} &\leq \P{\text{there is a conflicting pair of edges}}\\
&\leq \sum_{\substack{(e,f)\in E\times E\\ \abs {e\cap f}=1}} \P{(e,f)\text{ is a conflicting pair}} \\
&= \sum_{\substack{(e,f)\in E\times E\\ \abs {e\cap f}=1}} \P{\left(e\cap f \subseteq M\right)\cap \left(e\smallsetminus(e\cap f) \subseteq L\right) \cap \left(f\smallsetminus(e\cap f) \subseteq R\right)}\\
&= \sum_{\substack{(e,f)\in E\times E\\ \abs {e\cap f}=1}} \P{e\cap f \subseteq M} \cdot\P{e\smallsetminus(e\cap f) \subseteq L} \cdot\P{f\smallsetminus(e\cap f) \subseteq R}\\
&=\sum_{\substack{(e,f)\in E\times E\\ \abs {e\cap f}=1}} p \cdot x_{e\cap f}^{n-1}\cdot (1-x_{e\cap f})^{n-1} \\
&\leq \sum_{\substack{(e,f)\in E\times E\\ \abs {e\cap f}=1}} p \cdot \max_{x\in M}\left[x(1-x)\right]^{n-1} \\
&\leq (k\cdot 2^{n-1})^2 \cdot p \cdot \max_{x\in M}\left[x(1-x)\right]^{n-1}\\
&= k^2 \cdot 4^{n-1}\cdot p\cdot \frac{1}{4^{n-1}} = pk^2
\end{align*}
So $\P{\exists\text{ red edge}} \leq \P A + \P{A^c\cap \set{\exists\text{ red edge}}} \leq k(1-p)^n + kp^2$. This quantity is $<1$, whence $\P{\not\exists\text{ red edge}} >0$. This means that there is a coloring such that there is no red edge (there was no blue edge by construction). By definition, this is a $2-$coloring. So $m(n)$ must be greater than the number of edges of this graph, namely $k\cdot 2^{n-1}$.
\qed}

\begin{cor}
$m(n) > 2^{n-2}\cdot \sqrt{\frac{n}{\ln n}}$.
\end{cor}
\pf{
If $k=\frac{1}{2}\sqrt{\frac{n}{\ln n}}$ and $p = \frac{\ln n}{n}$. Then $1-p\leq e^{-p}\implies k(1-p)^n\leq ke^{-pn} = \frac{k}{n}$. Therefore $k^2p+k(1-p^n) \leq \frac{n}{4\ln n}\cdot\frac{\ln n}{n} + \frac{\sqrt{n}}{2n\sqrt{\ln n}} = \frac{1}{4} + \frac{1}{2\sqrt{n \ln n}} < 1$. By the above theorem, $m(n)>k\cdot 2^{n-1} = 2^{n-2}\cdot\sqrt{\frac{n}{\ln n}}$.\qed
}


\section{Example: Set Pairs}
\begin{thm}[Bollobas] Let $(A_i,B_i)$ for $1\leq i\leq h$ be pairs of subsets of $\Z$ satisfying that $\ds A_i\cap B_i=\varnothing\forall i, A_i\cap B_j\neq \varnothing \forall i\neq j$ and $\abs{A_i}=k,\abs{B_i}=\ell\forall i$. Then $h\leq {k+\ell\choose k}$. \\
(This is tight: Take $\abs{X}=k+\ell$ and $(A_i,B_i)$ are partitions of $X$ to disjoint sets of sizes $k,\ell$.)
\end{thm}

\begin{pf}
Order $\ds\bigcup_{i=1}^h A_i\cup B_i$ randomly. Let $E_i$ be the event that $A_i$ precedes $B_i$, that is, $\max A_i < \min B_i$. Note that $\P{E_i} = {k+\ell\choose k}^{-1}$. Also, events are pairwise disjoint, since if both $E_i,E_j$ occur together and (WLOG) $\max A_i \geq \max A_i$ then $\min B_i > \max A_i \geq \max A_j$ so $A_j\cap B_i=\varnothing$ which cannot happen. This means that $h\cdot {k+\ell\choose k}^{-1}=\sum_i\P{E_i} = \P{\bigcup_i E_i} \leq 1.$
\qed\end{pf}



\end{document}
