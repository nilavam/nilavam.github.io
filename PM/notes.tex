\documentclass[letterpaper,11pt]{article}

\usepackage{latexsym}
\usepackage[empty]{fullpage}
\usepackage{titlesec}
\usepackage{marvosym}
\usepackage[usenames,dvipsnames]{color}
\usepackage{verbatim}
\usepackage{enumitem}
\usepackage{hyperref}
\usepackage{fancyhdr}
\usepackage[english]{babel}
\usepackage{tabularx}
\usepackage{fontawesome5}
\usepackage{multicol}
\setlength{\multicolsep}{-3.0pt}
\setlength{\columnsep}{-1pt}
\input{glyphtounicode}
\usepackage{graphicx}
\usepackage{tikz}
\usepackage{libertinus}\usepackage[T1]{fontenc}
%----------FONT OPTIONS----------
% sans-serif
% \usepackage[sfdefault]{FiraSans}
% \usepackage[sfdefault]{roboto}
% \usepackage[sfdefault]{noto-sans}
% \usepackage[default]{sourcesanspro}

% serif
 %\usepackage{CormorantGaramond}
 %\usepackage{charter}

\def\mygap{-20pt}
\def\gap{-10pt}

\pagestyle{fancy}
\fancyhf{} % clear all header and footer fields
\fancyfoot{}
\renewcommand{\headrulewidth}{0pt}
\renewcommand{\footrulewidth}{0pt}

% Adjust margins
\addtolength{\oddsidemargin}{-0.4in}
\addtolength{\evensidemargin}{-0.4in}
\addtolength{\textwidth}{0.7in}
%\addtolength{\topmargin}{-0.5in}
%\addtolength{\textheight}{1in}
\setlength{\skip\footins}{8mm}

%\urlstyle{same}

\raggedbottom
\raggedright
\setlength{\tabcolsep}{0in}

% Sections formatting
\titleformat{\section}{
  \vspace{-4pt}\scshape\raggedright\Large\bfseries
}{}{0em}{}[\color{black}\titlerule \vspace{-5pt}]

% Ensure that generate pdf is machine readable/ATS parsable
\pdfgentounicode=1

%-------------------------
% Custom commands



\newcommand{\resumeItem}[1]{
  \item\small{
    {#1 \vspace{-2pt}}
  }
}

\newcommand{\classesList}[4]{
    \item\small{
        {#1 #2 #3 #4 \vspace{-2pt}}
  }
}

\newcommand{\resumeSubheading}[4]{
  \vspace{-1pt}
    \begin{tabularx}{\textwidth}[t]{X r}
      {\textbf{#1}} & \textbf{\small #2} \\
      {{\itshape\small #3} } & \textit{\small #4} \\
    \end{tabularx}\vspace{-2pt}
}



\newcommand{\resumeSubSubheading}[2]{
    \item
    \begin{tabular*}{0.97\textwidth}{l@{\extracolsep{\fill}}r}
      \textit{\small#1} & \textit{\small #2} \\
    \end{tabular*}\vspace{-7pt}
}

\newcommand{\resumeProjectHeading}[2]{
    \vspace{-6pt}\item
    \begin{tabular*}{1.001\textwidth}{l@{\extracolsep{\fill}}r}
      #1 & \textbf{\small #2}\\
    \end{tabular*}\vspace{-8pt}
}

\newcommand{\resumeSubItem}[1]{\resumeItem{#1}\vspace{-4pt}}

\renewcommand\labelitemi{$\vcenter{\hbox{\tiny$\bullet$}}$}
\renewcommand\labelitemii{$\vcenter{\hbox{\tiny$\bullet$}}$}

\newcommand{\resumeSubHeadingListStart}{\begin{itemize}[leftmargin=0.0in, label={}]}
\newcommand{\resumeSubHeadingListEnd}{\end{itemize}}
\newcommand{\resumeItemListStart}{\begin{itemize}[leftmargin=0.0in,label={}]}
\newcommand{\resumeItemListEnd}{\end{itemize}\vspace{-4pt}}
\newcommand{\eto}{\stackrel{=}{\to}}
\fontfamily{qcr}\selectfont 
%\usepackage[backend=bibtex]{biblatex}
\usepackage[
backend=biber,
style=alphabetic,%firstinits,
citestyle=ieee-alphabetic,
%natbib=true,
%uniquelist=false,
maxnames=10,
sorting=ynt
]{biblatex}
\addbibresource{writeup/article/refs.bib}
%\title{\vspace{-1cm}}
\title{The Probabilistic Method}
\usepackage{quiver}
\author{Instructor: Noga Alon\\Notes by Nilava Metya}
\date{Spring 2024}
\makeatletter
\renewcommand{\@chapapp}{Lecture}
\makeatother

\begin{document}

\maketitle
\tableofcontents

\chapter{01/30/2024}

\section{Philosophy}
Main philosophy of the probabilistic method: To prove existence of a structure (or a substructure of a given one), define a probability space of structures, and show that a random point in it satisfies the required properties with positive (often high) probability.

We will look at two examples today.

\section{Example: Graph Theory}

\begin{defn}[Ramsey numbers]
For $k,\l\geq 1$, let $r=r(k,\l)$ be the smallest integer, if there exists any, satisfying the following property: for every coloring of edges of $G=K_r$ (the complete graph on $r$ nodes) by \red~and \blue, either $\exists$ a  blue $K_k\subseteq G$ or a red $K_{\l}\subseteq G$.
\end{defn}
\begin{ex}
$r(3,3)=6$.
\end{ex}

A special case of Ramsey's theorem says that $\exists r(k,l)<\infty\forall k,l$. The proof, by induction (using Erd\"os-Szekeres theorem), gives $\ds r(k,\l)\leq\binom{k+\l-2}{k-1}$. In particular, $r(k,k) \leq \binom{2k-2}{k-1} <4^k$.

\begin{rmk}
The following are easy to observe: $r(k,\l)=r(l,k), r(1,\l)=1, r(2,\l)=\l$.
\end{rmk}

All the exactly known Ramsey numbers for $\l\geq k\geq 3$ are
$r(3,3)=6, r(3,4)=9, r(3,5)=14, r(3,6)=18, r(3,7)=23, r(3,8)=28, r(3,9)=36, r(4,4)=18, r(4,5)=25$. It is only known that $41\leq r(3,10)\leq 42, 36\leq r(4,6)\leq 40, 43\leq r(5,5)\leq 48$, and some similar bounds for other Ramsey numbers.

\begin{thm}[Erdos '47]
If $\ds\binom n k 2^{1-\binom k 2} < 1$ then $r(k,k)>n$. Therefore $r(k,k)\geq \left[1-o(1)\right]\frac{k}{e} 2^{\frac{k-1}2}$.
\end{thm}
\begin{pf}
Take the complete graph on $n$ labelled vertices $[n] = \set{1,\cdots,n}$. Color each edge $\set{i,j}$ (for $1\leq i<j\leq n$) randomly uniformly and independently either \red~or \blue. For fixed $K\subseteq[n]$ with $k=\abs K$, the probability that the graph induced by $K$ is monochromatic is $\ds{\color{red}2^{-\binom k 2}} + {\color{blue}2^{-\binom k 2}} = 2^{1-\binom k 2}$. So 
\begin{align*}
\ds\P{\exists \text{ such monochromatic }K} &\leq \sum_{\substack{K\subseteq [n]\\ \abs K = k}} \P{K \text{ induces a monochromatic graph}} \\
&= \binom n k 2^{1-\binom k 2} \stackrel{\text{given}}{<} 1.
\end{align*}
Therefore, $\ds\P{\not\exists \text{ such monochromatic }K} >0$. This means $r(k,k)>n$, which proves the first part.

Now, \begin{align*}
\binom n k 2^{1-\binom k 2} \leq 2 \left(\frac{en}{k}\right)^k\cdot 2^{-\binom k2} = 2\left(\frac{en}{2^{ \frac{k-1}{2}}\cdot k }\right)^k
\end{align*}
where the first inequality is due to $\ds\binom a b \leq \left(\frac{ea}{b}\right)^b$. If $\frac{en}{2^{ \frac{k-1}{2}}\cdot k }<1-\epsilon$ then for $k>k_0(\epsilon)$ for some $k_0(\epsilon)$, the RHS is $<1$. This implies that $r(k,k) \geq \left[1-o(1)\right]\frac{k}{e} 2^{\frac{k-1}2}$.\footnote{
Explanation for the last `implies': We note that for every $n$ satisfying the given condition, we have $r(k,k)>n$. Now for any $n<\left[1-\epsilon\right]\frac{k}{e} 2^{\frac{k-1}2}$, the condition is satisfied. Thus, $r(k,k)$ is more than all such $n$'s, which is written as $\left[1-o(1)\right]\frac{k}{e} 2^{\frac{k-1}2}$.
}
\qed\end{pf}

\rmk{
The lower bound was improved only by a factor of two since $1947$.\\
The upper bound was improved several times, last time in $2023$ by Campos, Griffiths, Morris, Sahasrabudhe to $(4-\epsilon)^k$, for an absolute constant $\epsilon>0$.\\
Open: Does $\lim r(k,k)^{1/k}$ exist (for USD 100)? If exists, find it (for USD 250).
}

\rmk{
Open problem: Find an explicit coloring showing $r(k,k)>1.0001^k$.
}

\rmk{This proof provides a randomized algorithm for finding a coloring that shows  $r(k,k) > \left\lfloor\sqrt{2^k}\right\rfloor$. But given such a coloring, we don't know how to efficiently check that $\not\exists$ a monochromatic $K_k$.
}

\section{Example: Dominating Sets}

\begin{defn}
If $G=(V,E)$ is a graph, we say $S\subseteq V$ is dominating if $\forall v\in V\smallsetminus S\exists u\in S$ such that $ \set{u,v}\in E$.
\end{defn}

\ex{
The set of bold vertices in \begin{tikzcd}
	\circ & \bullet & \circ \\
	\bullet & \circ
	\arrow[no head, from=1-1, to=1-2]
	\arrow[no head, from=1-2, to=1-3]
	\arrow[no head, from=1-2, to=2-2]
	\arrow[no head, from=1-1, to=2-1]
	\arrow[no head, from=2-1, to=2-2]
	\arrow[no head, from=1-1, to=2-2]
\end{tikzcd} form a dominating set.
}

\begin{thm}
If $G=(V,E)$ is a graph with $\abs V = n$ and minimum degree $\delta$, then it has a dominating set of size at most $\ds n\cdot \frac{1+\ln(1+\delta)}{1+\delta}$.
\end{thm}

\begin{pf}
Let $p=\frac{\ln(1+\delta)}{1+\delta}$. Clearly $p\in [0,1]$. Let $X\subseteq V$
 be a random subset of $V$ obtained by choosing each $v\in V$ to randomly and independently lie in $X$ with probability $p$. Since $X$ is not necessarily a dominating set, we can \textit{alter} it by $$Y_X\sett\set{v\in V\smallsetminus X\st \not\exists u\in X \text{ with } \set{u,v}\in E}.$$
By construction, $X\sqcup Y_X$ is a dominating set (note that they are disjoint).

Let's estimate the expected size of $X\cup Y_X$. First observe that $\E{\abs{X\cup Y_X}} = \E{\abs X + \abs {Y_X}}$ due to disjointness, and this is further equal to $\E{\abs X} + \E{\abs {Y_X}}$ by linearity of expectation.\\
$\abs X$ is a sum of independent indicators, one for each vertex which takes $1$ with probability $p$ and $0$ with probability $1-p$. So $\E{\abs X} = np$.\\
Note that $\P{v\in Y_X} = \P{v\notin X}\cdot \P{\text{no neighbor of } v \text{ is in } X} = (1-p)^{d_v} \leq (1-p)^{1+\delta} =\frac{1}{1+\delta}$ where $d_v$ is the degree of $v$ in $G$. Again $\abs{Y_X} = \sum_{v\in V} \pmb 1_{v\in Y_X}$ whence $\E{\abs {Y_X}} \leq \frac n {1+\delta}$. \\
This means $\E{\abs{X\cup Y_X}} \leq n\left[\frac{1+\ln(1+\delta)}{1+\delta}\right]$. Since the `average size' of a dominating set is less than or equal to the given quantity, $\exists$ a choice of $X$ such that $X\cup Y_X$ is a dominating set of size at most $\ds n\cdot \frac{1+\ln(1+\delta)}{1+\delta}$.
\qed\end{pf}

\rmk{We used \textit{linearity of expectation}: $\E{X+Y} = \E X + \E Y$. We also used \textit{alteration}: making a change after initial random choice, in this case we added $Y_X$ to $X$. (To be discussed more)}

\rmk{Here $\exists$ an efficient algorith to find such a dominating set. Start with $\varnothing$ and keep adding vertices that dominate maximum of yet non-dominated vertices.}

\rmk{Estimate is essentially that for $n\gg\delta\gg1$.}

\end{document}
