\documentclass[12pt]{amsart}
\usepackage[letterpaper, portrait, left = 1in, right = 1in, top = 1.2in, bottom=1.5in]{geometry} 
%\usepackage{setspace} \doublespacing
%\usepackage[letterpaper, portrait, margin=1.3in]{geometry}
\usepackage[table,xcdraw]{xcolor}
\usepackage{amssymb}
\usepackage{amsfonts}
\usepackage{longtable}
\usepackage{amsmath,amsthm}
\usepackage{enumitem}
\usepackage[utf8]{inputenc}
\usepackage{mathtools}
\usepackage{graphicx}
\usepackage{parskip}
\usepackage{multicol}
\usepackage{listings}
\usepackage[skip=0.25pt]{caption}
\usepackage[mathscr]{euscript}
\usepackage{quiver}
\setlength{\parindent}{0pt}
\usepackage{thm-restate}
\definecolor{vividburgundy}{rgb}{0.62, 0.11, 0.21}
\usepackage[driverfallback=hypertex,pagebackref=false,colorlinks,citecolor=vividburgundy]{hyperref}
\usepackage[capitalize]{cleveref}
%\usepackage[cmintegrals,cmbraces]{newtxmath}
%\usepackage{ebgaramond-maths}

%\usepackage{fourier}
%----------FONT OPTIONS----------
% sans-serif
%\usepackage[sfdefault]{FiraSans}
 %\usepackage[sfdefault]{roboto}
% \usepackage[sfdefault]{noto-sans}
%\usepackage[default]{sourcesanspro}

% serif
%\usepackage{CormorantGaramond}

%\usepackage{charter}
\usepackage[T1]{fontenc}
\usepackage{cleveref}
\definecolor{dg}{RGB}{10, 100, 10}
\setlength{\parindent}{0in}
\renewcommand{\qed}{$\hfill\blacksquare$}
% \newtheoremstyle{style}{2pt}{1pt}{\normalfont}{}{\bfseries}{\\}{0cm}{}
% \theoremstyle{style}
\newtheorem{lemma}{Lemma}[section]
% \newtheorem{lemma}{Lemma}
\newtheorem{thm}[lemma]{Theorem}
\newtheorem{prop}[lemma]{Proposition}
\newtheorem{cor}[lemma]{Corollary}
\newtheorem{conj}[lemma]{Conjecture}
\newtheorem{cl}[lemma]{Claim}
\newtheorem{rmk}{Remark}
\newtheorem{defn}[lemma]{Definition}
\newtheorem{qs}{Question}
\newtheoremstyle{styleS}{}{}{\color{dg}}{}{\color{dg}\bfseries}{. }{0cm}{}
\theoremstyle{styleS}
\newtheorem*{sol}{Solution}
\newtheoremstyle{style1}{}{}{\normalfont}{}{\bfseries}{. }{0cm}{}
\theoremstyle{style1}
\newtheorem{prob}{Problem}[section]
\newtheorem*{prb}{Problem}
\newtheoremstyle{style2}{1pt}{4pt}{\normalfont}{}{\itshape}{. }{0cm}{}
\theoremstyle{style2}
\newtheorem{ex}[lemma]{Example}
\newtheorem*{pf}{Proof}

\newcommand{\norm}[2]{
\left\lVert #1 \right\rVert_{#2}
}
\usepackage{mathrsfs}
%\usepackage[table,xcdraw]{xcolor}
\usepackage{booktabs}
\usepackage{tikz}
\usetikzlibrary{matrix}
\renewcommand{\l}{\ell}


\newcommand{\fA}{{\mathfrak{A}}}   \newcommand{\fB}{{\mathfrak{B}}}
\newcommand{\fC}{{\mathfrak{C}}}   \newcommand{\fD}{{\mathfrak{D}}}
\newcommand{\fE}{{\mathfrak{E}}}   \newcommand{\fF}{{\mathfrak{F}}}
\newcommand{\fG}{{\mathfrak{G}}}   \newcommand{\fH}{{\mathfrak{H}}}
\newcommand{\fI}{{\mathfrak{I}}}   \newcommand{\fJ}{{\mathfrak{J}}}
\newcommand{\fK}{{\mathfrak{K}}}   \newcommand{\fL}{{\mathfrak{L}}}
\newcommand{\fM}{{\mathfrak{M}}}   \newcommand{\fN}{{\mathfrak{N}}}
\newcommand{\fO}{{\mathfrak{O}}}   \newcommand{\fP}{{\mathfrak{P}}}
\newcommand{\fQ}{{\mathfrak{Q}}}   \newcommand{\fR}{{\mathfrak{R}}}
\newcommand{\fS}{{\mathfrak{S}}}   \newcommand{\fT}{{\mathfrak{T}}}
\newcommand{\fU}{{\mathfrak{U}}}   \newcommand{\fV}{{\mathfrak{V}}}
\newcommand{\fW}{{\mathfrak{W}}}   \newcommand{\fX}{{\mathfrak{X}}}
\newcommand{\fY}{{\mathfrak{Y}}}   \newcommand{\fZ}{{\mathfrak{Z}}}

\newcommand{\cA}{{\mathcal{A}}}   \newcommand{\cB}{{\mathcal{B}}}
\newcommand{\cC}{{\mathcal{C}}}   \newcommand{\cD}{{\mathcal{D}}}
\newcommand{\cE}{{\mathcal{E}}}   \newcommand{\cF}{{\mathcal{F}}}
\newcommand{\cG}{{\mathcal{G}}}   \newcommand{\cH}{{\mathcal{H}}}
\newcommand{\cI}{{\mathcal{I}}}   \newcommand{\cJ}{{\mathcal{J}}}
\newcommand{\cK}{{\mathcal{K}}}   \newcommand{\cL}{{\mathcal{L}}}
\newcommand{\cM}{{\mathcal{M}}}   \newcommand{\cN}{{\mathcal{N}}}
\newcommand{\cO}{{\mathcal{O}}}   \newcommand{\cP}{{\mathcal{P}}}
\newcommand{\cQ}{{\mathcal{Q}}}   \newcommand{\cR}{{\mathcal{R}}}
\newcommand{\cS}{{\mathcal{S}}}   \newcommand{\cT}{{\mathcal{T}}}
\newcommand{\cU}{{\mathcal{U}}}   \newcommand{\cV}{{\mathcal{V}}}
\newcommand{\cW}{{\mathcal{W}}}   \newcommand{\cX}{{\mathcal{X}}}
\newcommand{\cY}{{\mathcal{Y}}}   \newcommand{\cZ}{{\mathcal{Z}}}

\newcommand{\sA}{{\mathscr{A}}}   \newcommand{\sB}{{\mathscr{B}}}
\newcommand{\sC}{{\mathscr{C}}}   \newcommand{\sD}{{\mathscr{D}}}
\newcommand{\sE}{{\mathscr{E}}}   \newcommand{\sF}{{\mathscr{F}}}
\newcommand{\sG}{{\mathscr{G}}}   \newcommand{\sH}{{\mathscr{H}}}
\newcommand{\sI}{{\mathscr{I}}}   \newcommand{\sJ}{{\mathscr{J}}}
\newcommand{\sK}{{\mathscr{K}}}   \newcommand{\sL}{{\mathscr{L}}}
\newcommand{\sM}{{\mathscr{M}}}   \newcommand{\sN}{{\mathscr{N}}}
\newcommand{\sO}{{\mathscr{O}}}   \newcommand{\sP}{{\mathscr{P}}}
\newcommand{\sQ}{{\mathscr{Q}}}   \newcommand{\sR}{{\mathscr{R}}}
\newcommand{\sS}{{\mathscr{S}}}   \newcommand{\sT}{{\mathscr{T}}}
\newcommand{\sU}{{\mathscr{U}}}   \newcommand{\sV}{{\mathscr{V}}}
\newcommand{\sW}{{\mathscr{W}}}   \newcommand{\sX}{{\mathscr{X}}}
\newcommand{\sY}{{\mathscr{Y}}}   \newcommand{\sZ}{{\mathscr{Z}}}

\newcommand{\ta}{{\tilde{a}}}   \newcommand{\tb}{{\tilde{b}}}
\newcommand{\tc}{{\tilde{c}}}   \newcommand{\td}{{\tilde{d}}}
\newcommand{\te}{{\tilde{e}}}   \newcommand{\tf}{{\tilde{f}}}
\newcommand{\tg}{{\tilde{g}}}   
\newcommand{\ti}{{\tilde{i}}}   \newcommand{\tj}{{\tilde{j}}}
\newcommand{\tk}{{\tilde{k}}}   \newcommand{\tl}{{\tilde{l}}}
\newcommand{\tm}{{\tilde{m}}}   \newcommand{\tn}{{\tilde{n}}}
		         	\newcommand{\tp}{{\tilde{p}}}
\newcommand{\tq}{{\tilde{q}}}   \newcommand{\tr}{{\tilde{r}}}
\newcommand{\ts}{{\tilde{s}}}   
\newcommand{\tu}{{\tilde{u}}}   \newcommand{\tv}{{\tilde{v}}}
\newcommand{\tw}{{\tilde{w}}}   \newcommand{\tx}{{\tilde{x}}}
\newcommand{\ty}{{\tilde{y}}}   \newcommand{\tz}{{\tilde{z}}}

\newcommand{\red}{{\color{red}red}}
\newcommand{\blue}{{\color{blue}blue}}

\newcommand{\into}{\hookrightarrow}
\newcommand{\onto}{\twoheadrightarrow}
\newcommand\N{\ensuremath{\mathbb{N}}}

%\newcommand\L{\ensuremath{\mathbb{L}}}
\newcommand{\bP}{\mathbb{P}}
\newcommand\M{\ensuremath{\mathbb{M}}}
\newcommand\R{\ensuremath{\mathbb{R}}}
\newcommand\Z{\ensuremath{\mathbb{Z}}}
\renewcommand\O{\ensuremath{\emptyset}}
\newcommand\Q{\ensuremath{\mathbb{Q}}}
\newcommand\C{\ensuremath{\mathbb{C}}}
\newcommand{\K}{\ensuremath{\mathbb{K}}}
\newcommand\F{\ensuremath{\mathbb{F}}}
\newcommand{\aff}{\ensuremath{\mathbb{A}}}
\newcommand{\proj}{\ensuremath{\mathbb{P}}}
\newcommand{\dd}{\mathrm{d}}
\newcommand{\m}{\ensuremath{\mathfrak{m}}}
\newcommand{\p}{\ensuremath{\mathfrak{p}}}
\newcommand{\n}{\ensuremath{\mathfrak{n}}}
\renewcommand{\phi}{\varphi}
\renewcommand{\qedsymbol}{\ensuremath{\blacksquare}}
%\newcommand{\st}{\;|\;}
\newcommand{\st}{%
  \nonscript\;
  \ifnum\currentgrouptype=16
    \;\middle|\;
  \else
    \;|\;
  \fi
  \nonscript\;}
\newcommand{\ltr}{\par \noindent \framebox[1\width]{ $\implies$ } \hspace{.2cm}}
\newcommand{\rtl}{\par \noindent \framebox[1\width]{ $\impliedby$ } \hspace{.2cm} }
\newcommand{\abs}[1]{\left| #1 \right|}
\newcommand{\inner}[2]{\left\langle #1, #2 \right\rangle}
\newcommand{\E}[1]{\mathbb E\left[ #1 \right]}
\newcommand{\e}[1]{\exp\left( #1 \right)}
\renewcommand{\P}[1]{\mathbb P\left[ #1 \right]}
\newcommand{\Var}[1]{\text{Var}\left[ #1 \right]}
\newcommand*\circled[1]{\tikz[baseline=(char.base)]{
            \node[shape=circle,draw,inner sep=2pt] (char) {#1};}}
\newcommand{\ds}{\displaystyle}

\DeclareMathOperator{\sym}{Sym}
\DeclareMathOperator{\mds}{MDS}
\DeclareMathOperator{\Tor}{Tor}
\DeclareMathOperator{\Ext}{Ext}
\DeclareMathOperator{\adj}{adj}
\DeclareMathOperator{\Tr}{Tr}
\DeclareMathOperator{\GL}{GL}
%\DeclareMathOperator{\Tr}{Tr}
\DeclareMathOperator{\orbit}{Or}
\DeclareMathOperator{\stab}{Stab}
\DeclareMathOperator{\fix}{Fix}
\DeclareMathOperator{\re}{Re}
\DeclareMathOperator{\im}{Im}
\DeclareMathOperator{\ord}{Ord}
\DeclareMathOperator{\mspec}{mSpec}
\DeclareMathOperator{\spec}{Spec}
\DeclareMathOperator{\frob}{Frob}
\DeclareMathOperator{\id}{Id}
\DeclareMathOperator{\colim}{colim}
\DeclareMathOperator{\loc}{loc}
\DeclareMathOperator{\res}{Res}
\DeclareMathOperator{\rad}{rad}
\DeclareMathOperator{\Res}{Res}
\DeclareMathOperator{\diam}{diam}
\DeclareMathOperator{\arcsec}{arcsec}
\DeclareMathOperator{\arccot}{arccot}
\DeclareMathOperator{\len}{len}
\DeclareMathOperator{\area}{area}
\DeclareMathOperator{\vol}{vol}
\DeclareMathOperator{\ev}{ev}
\DeclareMathOperator{\sgn}{sgn}
\DeclareMathOperator{\supp}{supp}
\DeclareMathOperator{\diff}{d}
\DeclareMathOperator{\Dom}{Dom}
\DeclareMathOperator{\rk}{rank}
\renewcommand{\d}{\diff}
\let\oldend\endlinechar
\renewcommand{\endlinechar}{\oldend}
\newcommand{\open}{\underset{\text{open}}{\subset}}
\newcommand{\divides}{\mathbin{|}}
\newcommand{\set}[1]{\ensuremath{\left\{#1\right\}}}
\newcommand{\sett}{\coloneqq}
\newcommand*\isomap{%
  \xrightarrow{\raisebox{-0.9ex}[0ex][0ex]{$\sim$}}%
}
\renewcommand{\epsilon}{\varepsilon}
\newcommand{\fa}{~\forall~}
%\usepackage[nobottomtitles*]{titlesec}
\usepackage{titletoc}
%\titleformat{\section}[runin]
%{\normalfont\Large\bfseries}
%{}{0pt}{}%
%[\ifthenelse{\equal{\thesection}{0}}{\\\vspace*{0pt}}{\space\thesection}]
\newcommand{\sint}{\sin\theta}
\newcommand{\cost}{\cos\theta}
\newcommand{\tant}{\tan\theta}
\newcommand{\lb}{\left[}
\newcommand{\rb}{\right]}
\newcommand{\lp}{\left(}
\newcommand{\rp}{\right)}
\newcommand{\br}[1]{\lb#1\rb}
\newcommand{\pa}[1]{\lp#1\rp}
\usepackage{pdfpages}
\usepackage{fancyhdr}
	\pagestyle{fancyplain}
	\fancyhf{}
	\fancyhead[C]{\thepage}

%\fontfamily{qcr}\selectfont 
%\usepackage[backend=bibtex]{biblatex}
\usepackage[
backend=biber,
style=alphabetic,%firstinits,
citestyle=ieee-alphabetic,
%natbib=true,
%uniquelist=false,
maxnames=10,
sorting=ynt
]{biblatex}
%\addbibresource{writeup/article/refs.bib}
%\title{\vspace{-1cm}}
\title{\textbf{CONVEX AND CONIC OPTIMIZATION}\\ Homework $1$}
\usepackage{quiver}
\usepackage[nobottomtitles*]{titlesec}
\usepackage{titletoc}
\titleformat{\section}[runin]
  {\normalfont\Large\bfseries}
  {}{0pt}{}%
  [\ifthenelse{\equal{\thesection}{0}}{\\\vspace*{0pt}}{\space\thesection}]
\author{{\Large NILAVA METYA} \\ 
\href{mailto:nilava.metya@rutgers.edu}{nilava.metya@rutgers.edu}\\
\href{mailto:nm8188@princeton.edu}{nm8188@princeton.edu}}
\date{February $15$, $2024$}
\newcommand{\pb}{\section{Problem}~\par}
\newcommand{\soln}{\subsection*{Solution}}

\begin{document}

\maketitle

$\boxed{
B(x,r), B^o(x,r) \text{ respectively denote the closed and open balls of radius } r \text{ around }x.}$

\pb
Let $A$ be a real $m \times n$ matrix of rank $r$.
\begin{enumerate}[leftmargin=*]
\item 
\begin{enumerate}
\item Show that eigenvalues of $A^\top A$ are always nonnegative. (Hence singular values are well-defined as real, nonnegative scalars.)
\item Show that if $A$ is symmetric then the singular values of $A$ are the same as the absolute value of the eigenvalues of $A$.
\end{enumerate}
\item Consider the optimization problem: $$\min_{\substack{B\in\R^{m\times n}\\ \rk B\leq k} } \norm{A-B}{2}.$$
Here $\norm{\cdot}2$ denotes the spectral norm of a matrix defined as $\ds\norm C 2 = \max_{\norm x 2 = 1} \norm {Cx}2$. Show that the matrix $A_{(k)}$ is an optimal solution to the optimization problem above.
\item For $k = 40, 80, 120, 160$, use Matlab to compute $A_{(k)}$ as defined above. Report the value of $\norm{A - A(k)}F$ in each case. (Include your code for this part and the next.)
\item Use the commands {\texttt{subplot}} and \texttt{imshow} (you need to import the matplotlib package in Python) to produce on the same figure the original image, as well as your compressed images $A_{(k)}$ for $k = 40,80,120,160$. Label your subplots. In addition, produce two separate plots demonstrating $(i)\norm{A-A_{(k)}}F$ versus $k$, and $(ii)$``total savings'' versus $k$. Total savings is to be interpreted as the answer to the question: How many fewer numbers do you need in order to store $A_{(k)}$ than you did to store $A$? Explain why this number is equal to $mn-(n+m+1)k$. How much are you saving for $k=160$? 
\item Use the Matlab function \texttt{imwrite} or the Python command \texttt{image.save} to create two images from \texttt{imshow(A)} and \texttt{imshow(}$A_{(160)}$\texttt{)}. Can you tell them apart?
\end{enumerate}

\soln

\begin{enumerate}[leftmargin=*]
\item 
\begin{enumerate}
\item Say $\lambda\in\C$ be an eigenvalue of $A^\top A$ with eigenvector $u\in\C^n$. Then $A^\top Au=\lambda u\implies \lambda \norm{u}{2}^2 =\overline{u}^\top A^\top Au = (\overline{Au})^\top Au = (Au)^\dagger Au = \norm{Au}{2}^2$. The non-negativity of $\norm{Au}{2}^2$ and $\norm{u}{2}^2$ proves the non-negativity of $\lambda$. \\
Here, $\overline u$ is the coordinate-wise complex conjugate, and $(Au)^\dagger$ is the conjugate transpose. Note that we've used the fact that $\overline A = A$ because $A$ has real entries.
\item Suppose $A$ is an $n\times n$ real symmetric matrix. Then $A$ can be diagonalized via an orthonormal basis, that is, there is a real matrix $U$ such that $A=UDU^\top$, $U^\top U = UU^\top = \pmb 1_n$ and $D$ is a diagonal matrix comprising of eigenvalues of $A$. One can rearrange the columns of $U$ (the matrix of eigenvectors) such that their corresponding eigenvalues in $D$ are in order of decreasing magnitude. So one can assume without loss of generality that $D$ has entries $\lambda_1,\cdots,\lambda_n$ in this order where $\abs{\lambda_1}\geq \cdots \geq \abs{\lambda_n}$.\\
So $A^\top A = UD^\top U^\top UDU^\top = UD^2U^\top$, whence the eigenvalues of $A^\top A$ are $\lambda_1^2\geq \cdots\geq \lambda_n^2$. By what is mentioned in the assignment, $\sigma_i$ is the square root of the $i^{\text{th}}$ eigenvalue of $A^\top A$. In other words, $\sigma_i = \sqrt{\lambda_i^2} = \abs{\lambda_i}$ in this case.
\end{enumerate}
\item 

\item 

\item 

\item 

\end{enumerate}












\newpage
\pb

If ``True,” provide a proof. If ``False,” provide a counterexample and justify why your counterexample is valid.
\begin{enumerate}[leftmargin=*]
\item A point $\overline x \in \R^n$ is a local minimum of a quadratic (i.e., degree$-2$) polynomial $p : \R^n \to \R$ if and only if there are no descent directions at $\overline x$.
\item A point $\overline x \in \R^n$ is a local minimum of a cubic (i.e., degree$-3$) polynomial $p : \R^n \to \R$ if and only if there are no descent directions at $\overline x$.
\item Suppose $\Omega \subseteq \R^n$ is a closed convex set and $c$ is a vector in $\R^n$. Consider the problem of minimizing $c^\top x$ over $\Omega$. If this problem has a finite optimal value, then it has an optimal solution.
\end{enumerate}

\soln

\begin{enumerate}[leftmargin=*]
\item \textbf{True}.\\
We've already seen in class that if $\overline x$ is a local minima, then there is no descent direction of $p$ at $\overline x$. I'll prove that the converse is true for quadratic functions (please note: \textbf{we'll write $q$ for $\overline x$}). \\
%Let's actually prove the converse, namely that, if \\
If $p:\R^n\to \R$ is a quadratic polynomial, there is a symmetric real matrix $A\in \R^{n\times n}$, a real vector $b\in\R^n$ and a scalar $c\in \R$ such that $p(x) = x^\top Ax - 2b^\top x + c$. 

Consider the following statements for each point $x\in \R^n$ and each function $f:\R^n\to \R$:
\begin{itemize}[leftmargin=1in]
\item[$X(f,x):$] $x\in\R^n$ is not a local minima of $f:\R^n\to \R$.
\item[$Y(f,x):$] There is no descent direction of $f:\R^n\to \R$ at $x\in\R^n$.
\end{itemize}

\begin{lemma}\label{diag}
If $\nabla^2 p(x) = A = \text{diag}(\lambda_1,\cdots,\lambda_n)$ is diagonal (where $p$ is the above quadratic polynomial), there is no $q\in\R^n$ such that both $X(p,q),Y(p,q)$ are true.
\end{lemma}
\begin{pf}
For the sake of contradiction, assume $\exists q\in\R^n$ is such that both $X(p,q),Y(p,q)$ are true. $Y(p,q)\implies \nabla p(q) = 0\implies Aq=b\implies \lambda_iq_i=b_i\forall i$.

\textit{Case $1$: Some eigenvalue of $A$ is negative.} WLOG, say $\lambda_1<0$. Then we claim that $e_1$ is a descent direction. Indeed, if $\alpha\in (0,1)$ then \begin{align*}
f(q+\alpha e_1)-f(q) &= \alpha^2 e_1^\top Ae_1 + 2\alpha q^\top Ae_1 - 2\alpha b^\top e_1 \\
&= \alpha^2 \lambda_1 + 2\alpha(\lambda_1 q_1 - b_1) \\
&= \alpha^2 \lambda_1 < 0.\end{align*}
[So $\overline \alpha=1$ here]. This proves $Y(p,q)$ false.

\textit{Case $2$: All eigenvalues of $A$ are non-negative.} Then $A\succeq 0$. For any $y\in\R^n$
\begin{align*}
p(q+y)-p(q) &= y^\top A y + 2q^\top Ay -2b^\top y \\
&= y^\top A y + 2y^\top Aq -2b^\top y \\
&= y^\top A y + 2y^\top b -2b^\top y \\
&= y^\top Ay \geq 0\end{align*} where the last inequality is true because $A\succeq 0.$
Taking $y=a-q$ gives $p(a)\leq (q)$, and this inequality holds for all $a\in\R^n$ (because it holds for every $y\in\R^n$). Therefore, $q$ is a local minima of $p$, proving $X(p,q)$ false.

These two cases are exhaustive and prove that $X(p,q), Y(p,q)$ cannot be simultaneously true.
\qed\end{pf}

\begin{lemma}
There is no $q\in\R^n$ such that both $X(p,q),Y(p,q)$ are true, where $p$ is the aforementioned quadratic polynomial.
\end{lemma}
\begin{pf}
Since $A$, is real symmetric, there is an orthogonal matrix $U$ (so $U^\top U=UU^\top=\id_n$) such that $A=UDU^\top$ where $D=\text{diag}(\lambda_1,\cdots,\lambda_n)$ is the diagonal comprising of eigenvalues of $A$. Define $g(x) = p(Ux)$. Then \begin{align*}
g(x) &= x^\top U^\top (UDU^\top) Ux - 2b^\top Ux + c\\
&= x^\top D x - 2 \tb^\top x + c
\end{align*}
where $\tb = U^\top b$. it satisfies that $\nabla^2 g(x)$ is diagonal for every $x\in\R^n$, so by \Cref{diag} there is no point $x$ such that $X(g,x), Y(g,x)$ are both true. 

Fix a $q\in\R^n$. Then at least one of $X(g,U^\top q)$ or $Y(g,U^\top q)$ is false. Note $g(U^\top q) = p(q)$. \\
\textit{If $X(g,U^\top q)$ is false,} then $U^\top q$ is a local minima of $g$. So $\exists\delta >0$ such that if $\norm{y-U^\top q}{} < \delta$ then $p(Uy)=g(y) \leq g(U^\top q) = p(q)$. So if $y\in \R^n$ satisfies $\norm{y-q}{} < \delta$ then using the fact that $\norm{U^\top y-U^\top q}{} = \norm{y-q}{}<\delta$ (orthogonal matrices preserve the $2-$norm) we get $p(y)=p(U(U^\top y)) \leq p(q)$ whence $q$ is a local minima for $p$, thus $X(p,q)$ is false.\\
\text{If $Y(g,U^\top q)$ is false,} then there is a (descent) direction $d$ and $\overline\alpha>0$ such that $g(U^\top q+\alpha d) < g(U^\top q) = p(q)$. Noting that $g(U^\top q+\alpha d) = p(q + \alpha (Ud))$ and setting $\td \sett Ud$ gives $p(q + \alpha \td) < p(q)\forall \alpha\in(0,\alpha)$, whence $p$ has a descent direction $\td$ at $q$. This means $Y(p,q)$ is false.

Since at least one of $X(g,U^\top q), Y(g,U^\top q)$ is false, we conclude that at least one of $X(p,q), Y(p,q)$ is false.
\qed\end{pf}




\item \textbf{False}.\\
It is true (as seen in class) that if $p$ has a local minima at $\overline x$, then there is no descent direction of $p$ at $\overline x$. We show a counterexample for the converse. Consider the function $p:\R^2\to \R$ given by $p(x_1,x_2)=x_2(x_2-x_1^2)$. We prove that $p$ has no descent direction at $\overline x = (0,0)$, yet isn't a local minima. \textit{A wider class of examples can be found from \Cref{gen3} in the \href{sec3app}{Appendix} of this problem.}

Let's see why it's \textit{not a local minima}. Indeed $p(\overline x) = 0$ and the sequence of points $\ds\set{q_k=\left(\frac{1}{k},\frac{1}{2k^2}\right)}_{k\in\N}$ converge to $\overline x = (0,0)$ (because their norm $\frac{1}{k}\cdot \sqrt{1+\frac{1}{2k^2}}$ converges to $0$ as $k\to\infty$) and yet $p(q_k) = \frac{-1}{4k^4}<0\forall k\in\N$, proving that there is no ball around $\overline x$ where all points have functional value (under $p$) at least $p(\overline x) = 0$.

Now we show that there is \textit{no descent direction of $p$ at $\overline x$}. Suppose $(a,b)$ is a descent direction. Define $x_\alpha = (\alpha a,\alpha b)$ for $\alpha\in \R$. Then there is some $\overline \alpha>0$ such that $\alpha\in(0,\overline \alpha)\implies p(x_\alpha) = \alpha b(\alpha b-\alpha^2a^2)<0$. By positivity of $\alpha$, this is equivalent to saying $b(b-\alpha a^2)<0$. Note that $b\neq 0$ because the inequality is strict. If $b<0$ then $b-\alpha a^2 < -\alpha x^2 \leq 0$ implying that $p(x_\alpha) > 0$. If $b>0$, then taking $\alpha\in \left(0,\frac{1}{2}\min\set{\frac{b}{a^2},\overline \alpha}\right)$ we get $b-\alpha a^2 > b-\frac{b}{a^2}\cdot a^2 = 0$ whence $p(x_\alpha)>0$. This contradicts the fact that $(a,b)$ is a descent direction.

\item \textbf{False}.\\
Consider the set $\Omega = \set{(x_1,x_2)\in \R^2\st x_1\geq0, x_1x_2\geq 1}$ with $c=(0,1)$. \\
$\Omega$ is closed because $\Omega$ is the intersection of inverse image of $[1,\infty)$ under the function $(x_1,x_2)\to x_1x_2$) and the inverse image of $[0,\infty)$ under the function $(x_1,x_2)\to x_1$), and both of these are continuous functions (we've used the fact that closed sets go to closed sets under inverses of continuous functions and that the intersection of two closed sets is closed). \\
This is also convex, which is seen as follows. Let $(a,b),(u,v)\in \Omega, \lambda\in[0,1]$. Note that $u,v\geq 0$. Then taking $\mu=1-\lambda\in[0,1]$ gives $\lambda a + \mu u \geq 0$ because $\lambda,a,u\geq 0$, and $(\lambda a + \mu u)(\lambda b + \mu v) = \lambda^2 a + \lambda\mu(bu+av) + \mu^2 v \geq \lambda^2 + \lambda\mu\left(\frac{u}{a} + \frac{a}{u}\right) + \mu^2 \ge \lambda^2 + 2\lambda\mu + \mu^2 = 1$. So $\lambda (a,b) + (1-\lambda)(u,v) \in\Omega$.\\
Then the objective function is just $f(x_1,x_2) = (0,1)^\top(x_1,x_2) = x_2$. This gives $\ds\inf_{(x_1,x_2)\in\Omega} x_2 = 0$ but there is no $(x_1,x_2)$ such that $x_2=0$ because of the given conditions. So optimal value exists, but optimal solution doesn't.
\end{enumerate}

\subsection*{Appendix}\label{sec3app}

\begin{thm}\label{gen3}
Let $f,g:\R\to \R$ be $\sC^1$ functions such that  such that \begin{itemize}
\item $\exists\epsilon>0$ such that $(f-g)^{-1}(0)\cap (0,\epsilon)=\varnothing$,
%\item $f'',g''$ exists on $B^o(0,\epsilon)$,
\item $f(0)=g(0)=0$, 
\item $f'(0)=g'(0)$ (call it $d$), and
\item $\exists \delta > 0$ such that $d$ does not lie strictly between $\frac{f(t)}{t}$ and $\frac{g(t)}{t}$ for every $t\in(0,\epsilon)$,
\end{itemize}
then $h:\R^2\to\R$ given by $h(x,y) = (y-f(x))(y-g(x))$ has no descent direction at $(0,0)$ and $(0,0)$ is not a local minima of $h$.
\end{thm}

\begin{pf}
Suppose $(a,b)$ is a descent direction at $(0,0)$. So there is some $\overline\alpha$ such that $h(\alpha a,\alpha b) < h(0,0) \forall \alpha\in(0,\overline\alpha)$. We can assume, WLOG, that $0<a^2+b^2<\epsilon$ (for example, by replacing $(a,b)$ with $(\lambda a, \lambda b)$ such that $\lambda<\frac{\sqrt\epsilon}{\overline \alpha\sqrt{a^2+b^2}}$). One must have $f>g$ or $g>f$ on $(0,\epsilon)$ because otherwise, by continuity, there would be a non-zero point in $(0,\epsilon)\subset B^o(0,\epsilon)$ which is a zero for the function $f-g$. WLOG, assume $f>g$ on $(0,\epsilon)$. This implies $0 > h(a,b) = (b-f(a))(b-g(a))$, whence $g(a) < b < f(a)$. In fact, it is true that $0>h(\alpha a, \alpha b) = (\alpha b-f(\alpha a))(\alpha b-g(\alpha a))\forall \alpha\in (0,\overline \alpha)$ by definition. From this it follows that $a\neq 0$ (because if $a=0$ then $h(0,\alpha b) = (\alpha b)^2 > 0$). We thus have $g(\alpha a)<\alpha b < f(\alpha a)$ which gives that $b = da$ by using the sandwich theorem. Then note that if $\alpha = \frac{1}{2}\min\set{\frac{\delta}{a},\overline \alpha}$ we have $h(\alpha a, \alpha b) = (\alpha b-f(\alpha a))(\alpha b - g(\alpha a)) = (\alpha ad-f(\alpha a))(\alpha ad - g(\alpha a)) = t^2 \left(d-\frac{f(t)}{t}\right) \left(d-\frac{g(t)}{t}\right)$ where $t = \alpha a < \delta$ so this quantity is $\ge0$ by the assumption in the last bullet point, however this contradicts the fact that $h(\alpha a, \alpha b) < 0$.

Now we show that $(0,0)$ not a local minima. Since $f>g$ on $(0,\epsilon)$, one can choose points $x_n=\frac{1}{n}$ and $y_n\in (g(x_n),f(x_n))$ and observe that $\lim y_n = 0$ by continuity of $f,g$ and sandwich theorem. Thus, for every $\varepsilon>0\exists N\in \N$ such that $k\geq N\implies \abs{y_k} < \varepsilon$. But $h(x_n,y_n) = (y_n-f(x_n))(y_n-g(x_n)) < 0$ by choice of $y_n$. This proves that $(0,0)$ cannot be a local minima of $h$, because $(x_n,y_n)\to (0,0)$ by each $h(x_n,y_n)<0$.
\qed\end{pf}





\newpage
\pb
\begin{enumerate}[leftmargin=*]
\item Let $Q\in S^{n\times n}$ and assume $Q\succ 0$. Show that $\ds f(x) = \sqrt{x^\top Qx}$ is a norm.
\item Show that $Q^{-1}$ exists and is positive definite. Show that the dual norm of $f$ is given by $\ds f(x) = \sqrt{x^\top Q^{-1}x}$.
\item Let $A \in\R^{m\times n}$. Prove the following expression for its induced $2-$norm: $\ds\norm A2 = \sqrt{\lambda_{\max}(A^\top A)}$.
\end{enumerate}

\soln
$Q$ is symmetric positive definite, whence $Q=UDU^\top$ with $D$ being a diagonal matrix of positive diagonal entries (say $\lambda_1^2, \cdots,\lambda_n^2$), and $U$ a real orthogonal matrix. For a vector $x = (x_1,\cdots,x_n)$, we denote $\tilde x=U^\top x$ and its coordinates by $\tx=(\tx_1,\cdots,\tx_n)$.
\begin{enumerate}[leftmargin=*]
\item Note that $x^\top Qx = \tx^\top D \tx = \sum_i \lambda_i^2\tx_i^geq$ for any $x\in\R^n$. So \begin{itemize}[leftmargin=*]
\item (Positivity) $x^\top Q x \geq 0$ by definition of positive semi-definiteness (which is implied by $Q\succ 0$). And if $x \neq 0$ then $x^\top Qx > 0$ because $Q\succ 0$; so that if $x^\top Q x=0$ then $x=0$.
\item (Triangle inequality) Observe $f(x+y)^2= \sum_i \lambda_i^2(\tx_i+\ty_i)^2 = f(x)^2 + f(y)^2 + 2\sum_i \lambda_i^2\tx_i\ty_i$.
Thus \begin{align*}
\left(\sum_i\lambda_i^2\tx_i^2\right)\left(\sum_j\lambda_j^2\ty_j^2\right)
&=\sum_{1\leq i,j\leq n} \lambda_i^2\lambda_j^2 \left(\tx_i^2\ty_j^2\right) \\
&= \sum_{1\leq i<j\leq n} \lambda_i^2\lambda_j^2 \left(\tx_i^2\ty_j^2+\tx_j^2\ty_i^2\right)\\
&\geq 2\sum_{1\leq i<j\leq n} \lambda_i^2\lambda_j^2 (\tx_i\ty_j\tx_j\ty_i)\\
&= \sum_{1\leq i,j\leq n} \lambda_i^2\lambda_j^2 (\tx_i\ty_j\tx_j\ty_i)\\
&= \left(\sum_i\lambda_i^2\tx_i\ty_i\right)\left(\sum_j\lambda_j^2\tx_j\ty_j\right)\\
\implies f(x)^2f(y)^2 = \left(\sum_i\lambda_i^2\tx_i^2\right)\left(\sum_j\lambda_j^2\ty_j^2\right) &\leq \left(\sum_i\lambda_i^2\tx_i\ty_i\right)^2\\
\implies 2f(x)f(y)  &\leq 2\sum_i\lambda_i^2\tx_i\ty_i = f(x+y)^2-f(x)^2-f(y)^2\\
\implies \left[f(x) + f(y)\right]^2 &\leq f(x+y)^2\\
\implies f(x) + f(y) &\leq f(x+y).
\end{align*}
(One can alternately prove Cauchy-Schwarz inequality for real inner products, which it is in this case due to positive definiteness, from which the above inequality follows --- without the eigendecomposition)
\item (Homogeneity) $f(\lambda x) = \sqrt{\lambda^2 x^\top Qx} = \abs{\lambda}\sqrt{x^\top Qx} = \abs\lambda f(x)$ for any $\lambda\in\R,x\in\R^n$.
\end{itemize}
\item  Note that $D$ is invertible, its inverse is simply given by the diagonal matrix with diagonal entries $\frac{1}{\lambda_i}>0$. Taking $P = UD^{-1}U^\top$ gives $PQ = UD^{-1}U^\top UDU^\top = UD^{-1}DU^\top = UU^\top = \pmb 1_n$ and this proves that $P$ is the inverse of $Q$ because $Q$ is a square matrix. 

Denote $\sqrt{Q}\sett U \sqrt D U^\top$ where $D=\text{diag}(\sqrt{\lambda_1},\cdots,\sqrt{\lambda_n})$ and observe that it is symmetric, invertible and satisfies $\sqrt Q^2 = Q$, because $\sqrt{D}$ is invertible (positive eigenvalues) and $\sqrt D^2=D$.
Recall that the dual norm is given by $\left(\norm{x}{*}=\right)\norm{x}{*,f} = \sup\set{x^\top y : f(y)\leq 1}$. We start with the observation that $f(y) = \sqrt{y^\top Q y} = \sqrt{y^\top \sqrt Q \sqrt Q y} = \sqrt{y^\top \sqrt Q^\top \sqrt Q y} = \norm{\sqrt Qy}{2}$. 
Note that $y=\frac{Q^{-1}x}{\sqrt{x^\top Q^{-1}x}}$ satisfies $f(y)^2 = y^\top Q y = \frac{x^\top Q^{-1} Q Q^{-1}x}{x^\top Q^{-1}x} = \frac{x^\top Q^{-1}x}{x^\top Q^{-1}x} = 1$ and $y^\top x= \frac{x^\top Q^{-1}x}{\sqrt{x^\top Q^{-1}x}} = \sqrt{x^\top Q^{-1}x}$. This shows that $\norm{x}{*} \geq \sqrt{x^\top Q^{-1}x}$. But Cauchy Schwarz inequality gives 
$$x^\top y = \left(\left(\sqrt Q\right)^{-1} x\right)^\top\left(\sqrt Qy\right)\leq \norm{\left(\sqrt Q\right)^{-1}x}{2}\norm{\sqrt Qy}{2} \leq \norm{\left(\sqrt Q\right)^{-1}x}{2}=\sqrt{x^\top Q^{-1}x}.$$ This proves that $\norm{x}* = \sqrt{x^\top Q^{-1}x}$.

\item $A\in\R^{n\times n}$. $\ds\norm A 2 ^2 = \max_{\norm x 2 = 1}\norm{Ax}{2}^2 = \max_{\norm x 2 = 1} x^\top (A^\top A)x$. Say $\mu_i$ are the eigenvalues of $A^\top A$ with $\mu_1\geq\cdots\geq \mu_n$. By the previous problem, $\mu_i\geq 0$. And by the calculation shown above, $x^\top A^\top Ax = \sum_i\mu_i\tx_i^2$. Taking $\tx = (1,0,\cdots,0) \eqqcolon \tx_0$ gives $x^\top A^\top Ax = \mu_1$, so $\boxed{\norm A 2 ^2 \ge \mu_1}$ because $\norm{\tx_0}2=1$. Note that we can independently choose $\tx$ because $x,\tx$ are different only by unitary transformation (both invertible and preserves norm: $\norm{Ux}{}^2 = x^\top U^\top U x = x^\top x = \norm x{}^2$).
\\
The observation that $\max\set{\mu_i}=\mu_1$ gives $\norm{Ax}{2}^2 = \sum_i\mu_i\tx_i^2 \leq \sum_i \mu_1 \tx_i^2 = \mu_1$ whence $\boxed{\norm A 2 ^2 \leq \mu_1}$.
\end{enumerate}













\newpage
\pb
Prove or disprove the following statements. All matrices are \textit{symmetric, $n \times n$, and with real entries}.
\begin{enumerate}[leftmargin=*]
\item Suppose $A \succeq 0$. Then the largest entry in absolute value of $A$ must be on the diagonal.
\item If $A \succeq 0$ and $\Tr(A) = 0$, then $A = 0$.
\item If $A \succeq 0, B \succeq 0$, and $A + B = 0$, then $A = B = 0$.
\item If $A \succeq 0, B \succeq 0$, and $AB = 0$, then $A = 0$ or $B = 0$.
\item Suppose $x^\top Ax\ge0$ for all $x\ge0$, then either $A\ge0$ or $A\succeq0$. \end{enumerate}
\soln 

\begin{enumerate}[leftmargin=*]
\item \textbf{True}.\\
For the sake of contradiction, assume there are indices $i< j$ such that $\abs{A_{ji}}=\abs{A_{ij}}>\abs{A_{kk}}\forall 1\leq k\leq n$. First we see that each $e_k^\top Ae_k = A_{kk}\geq 0$ by positive semi-definiteness. Next $0\leq (e_i+e_j)^\top A (e_i+e_j) = A_{ii} + 2A_{ij} + A_{jj}\implies A_{ii}+ A_{jj} \ge-2A_{ij}$. But, $A_{ii}, A_{jj} < \abs{A_{ij}}\implies A_{ii}+A_{jj} < 2 \abs{A_{ij}}$. Putting these together, we have $2\abs{A_{ij}} > A_{ii} + A_{jj} \ge -2A_{ij}$ which suggests that $A_{ij}\ge 0$ and thus $\abs{A_{ij}}=A_{ij}$. Now $(e_i-e_j)^\top A(e_i-e_j) = A_{ii} + A_{jj} - 2A_{ij} = A_{ii} + A_{jj} - 2\abs{A_{ij}} < 0$, contradicting that $A$ is positive semi-definite.
\item \textbf{True}.\\
$A\succeq 0$ implies that all eigenvalues are non-negative. Recall that trace is simply the sum of eigenvalues. But if their sum (of non-negative numbers) is zero, the only way that's possible is that all of them are zero. So $A$ is the conjugate of the zero matrix (because $A = UDU^\top$ for orthogonal $U$ and diagonal $D$ of eigenvalues, which is the $0$ matrix in this case).
\item \textbf{True}.\\
$A=-B\implies A,B$ are simultaneously diagonalizable (because each of them are). So $UAU^\top = -UBU^\top$ is diagonal. This means that if $\lambda_i$'s are the eigenvalues of $A$, the eigenvalues of $B$ are $-\lambda_i$'s. Positive semi-definiteness implies that $\lambda_i\geq 0, -\lambda_i\geq 0\forall i$ and thus $\lambda_i = 0\forall i$. So $UAU^\top = -UBU^\top$ is the $0$ matrix. This means $A=B=0$.
\item \textbf{False}.\\
For a concrete example, take $A = \begin{bmatrix}1&0\\0&0\end{bmatrix}, B = \begin{bmatrix}0&0\\0&1\end{bmatrix}$ are both symmetric positive semi-definite satisfying $AB=0$, but neither of $A,B$ is $0$. One can find such an example for each dimension by taking $A=\text{diag}(1,0,\cdots,0,0), B=\text{diag}(0,0,\cdots,0,1)$.
\item \textbf{False}.\\
Consider $A = \begin{bmatrix}1&-1&0\\-1&1&1\\0&1&0\end{bmatrix}$. First observe that $A = \begin{bmatrix}0&0&0\\0&0&1\\0&1&0\end{bmatrix} + \begin{bmatrix}1&-1&0\\-1&1&0\\0&0&0\end{bmatrix}$ where the first summand is entry-wise non-negative and the second summand is symmetric positive semi-definite (with eigenvalues $0,2$). This shows that $x^\top Ax\geq 0\forall x\geq 0$. But neither does $A$ have all non-negative entries, nor is $A$ positive semi-definite (because $(e_2-e_3)^\top A(e_2-e_3) = -1$).
\end{enumerate}


\end{document}
