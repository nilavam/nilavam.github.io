\documentclass[letterpaper,11pt]{article}

\usepackage{latexsym}
\usepackage[empty]{fullpage}
\usepackage{titlesec}
\usepackage{marvosym}
\usepackage[usenames,dvipsnames]{color}
\usepackage{verbatim}
\usepackage{enumitem}
\usepackage{hyperref}
\usepackage{fancyhdr}
\usepackage[english]{babel}
\usepackage{tabularx}
\usepackage{fontawesome5}
\usepackage{multicol}
\setlength{\multicolsep}{-3.0pt}
\setlength{\columnsep}{-1pt}
\input{glyphtounicode}
\usepackage{graphicx}
\usepackage{tikz}
\usepackage{libertinus}\usepackage[T1]{fontenc}
%----------FONT OPTIONS----------
% sans-serif
% \usepackage[sfdefault]{FiraSans}
% \usepackage[sfdefault]{roboto}
% \usepackage[sfdefault]{noto-sans}
% \usepackage[default]{sourcesanspro}

% serif
 %\usepackage{CormorantGaramond}
 %\usepackage{charter}

\def\mygap{-20pt}
\def\gap{-10pt}

\pagestyle{fancy}
\fancyhf{} % clear all header and footer fields
\fancyfoot{}
\renewcommand{\headrulewidth}{0pt}
\renewcommand{\footrulewidth}{0pt}

% Adjust margins
\addtolength{\oddsidemargin}{-0.4in}
\addtolength{\evensidemargin}{-0.4in}
\addtolength{\textwidth}{0.7in}
%\addtolength{\topmargin}{-0.5in}
%\addtolength{\textheight}{1in}
\setlength{\skip\footins}{8mm}

%\urlstyle{same}

\raggedbottom
\raggedright
\setlength{\tabcolsep}{0in}

% Sections formatting
\titleformat{\section}{
  \vspace{-4pt}\scshape\raggedright\Large\bfseries
}{}{0em}{}[\color{black}\titlerule \vspace{-5pt}]

% Ensure that generate pdf is machine readable/ATS parsable
\pdfgentounicode=1

%-------------------------
% Custom commands



\newcommand{\resumeItem}[1]{
  \item\small{
    {#1 \vspace{-2pt}}
  }
}

\newcommand{\classesList}[4]{
    \item\small{
        {#1 #2 #3 #4 \vspace{-2pt}}
  }
}

\newcommand{\resumeSubheading}[4]{
  \vspace{-1pt}
    \begin{tabularx}{\textwidth}[t]{X r}
      {\textbf{#1}} & \textbf{\small #2} \\
      {{\itshape\small #3} } & \textit{\small #4} \\
    \end{tabularx}\vspace{-2pt}
}



\newcommand{\resumeSubSubheading}[2]{
    \item
    \begin{tabular*}{0.97\textwidth}{l@{\extracolsep{\fill}}r}
      \textit{\small#1} & \textit{\small #2} \\
    \end{tabular*}\vspace{-7pt}
}

\newcommand{\resumeProjectHeading}[2]{
    \vspace{-6pt}\item
    \begin{tabular*}{1.001\textwidth}{l@{\extracolsep{\fill}}r}
      #1 & \textbf{\small #2}\\
    \end{tabular*}\vspace{-8pt}
}

\newcommand{\resumeSubItem}[1]{\resumeItem{#1}\vspace{-4pt}}

\renewcommand\labelitemi{$\vcenter{\hbox{\tiny$\bullet$}}$}
\renewcommand\labelitemii{$\vcenter{\hbox{\tiny$\bullet$}}$}

\newcommand{\resumeSubHeadingListStart}{\begin{itemize}[leftmargin=0.0in, label={}]}
\newcommand{\resumeSubHeadingListEnd}{\end{itemize}}
\newcommand{\resumeItemListStart}{\begin{itemize}[leftmargin=0.0in,label={}]}
\newcommand{\resumeItemListEnd}{\end{itemize}\vspace{-4pt}}
\usepackage{wrapfig}
\usepackage{multicol}
\usepackage[breakable]{tcolorbox}
%\fontfamily{qcr}\selectfont 
%\usepackage[backend=bibtex]{biblatex}
\usepackage[
backend=biber,
style=alphabetic,%firstinits,
citestyle=ieee-alphabetic,
%natbib=true,
%uniquelist=false,
maxnames=10,
sorting=ynt
]{biblatex}
%\addbibresource{writeup/article/refs.bib}
%\title{\vspace{-1cm}}
\title{\textbf{CONVEX AND CONIC OPTIMIZATION}\\ Homework $6$}
\usepackage{quiver}
\usepackage[nobottomtitles*]{titlesec}
\usepackage{titletoc}
\titleformat{\section}[runin]
  {\normalfont\Large\bfseries}
  {}{0pt}{}%
  [\ifthenelse{\equal{\thesection}{0}}{\\\vspace*{0pt}}{\space\thesection}]
\author{{\Large NILAVA METYA} \\ 
\href{mailto:nilava.metya@rutgers.edu}{nilava.metya@rutgers.edu}\\
\href{mailto:nm8188@princeton.edu}{nm8188@princeton.edu}}
\date{May $2$, $2024$}
\newcommand{\pb}{\section{Problem}~\par}
\newcommand{\soln}{\subsection*{Solution}}
\newcommand{\conv}{\text{conv}}
\newcommand{\epi}{\text{epi}}
\usepackage{pdfpages}
\usepackage{fancyhdr}
	\pagestyle{fancyplain}
	\fancyhf{}
	\fancyhead[R]{\thepage}
\setlength{\columnsep}{1.2cm}
\newcommand{\fa}{~\forall}
\begin{document}

\maketitle

\pb

\begin{enumerate}[leftmargin=*]
\item Suppose you had a blackbox that given a 3SAT instance would tell you whether it is satisfiable or not. How can you make polynomially many calls to this blackbox to find a satisfying assignment to any satisfiable instance of 3SAT?
\item Suppose you had a blackbox that given a graph $G$ and an integer $k$ would tell you whether $G$ has a stable set of size larger or equal to $k$. How can you make polynomially many calls to this blackbox to find a maximum stable set of a given graph?
\end{enumerate}


\soln

\begin{enumerate}[leftmargin=*]
\item Denote the blackbox by $f$. So $f$ takes in a formula (in three variables) and outputs $1$ if satisfiable, and $0$ otherwise. \\
Let $S$ be a formula in CNF, with variables $x_{1},\cdots,x_{n}$. Treat $S$ as a polynomial in $x_{i}$'s. Assume $S$ is satisfiable, i.e., there are values $a_{1},\cdots,a_{n}\in\set{0,1}$ such that $S(a_{1},\cdots,a_{n})=1$. Often we denote $\pmb a = (a_{1},\cdots,a_{n})$. We will make $n$ calls to the blackbox.
\begin{cl}
For each $i\in[n]$, at least one of $f(x_{i}\land S)$ or $f(\overline{x_{i}}\land S)$ is $1$.
\end{cl}
\begin{proof}
If $a_{i}=1$ then $\left(x_{i}\land S\right)(\pmb a) = 1\cdot S(\pmb a) = 1$. 
If $a_{i}=0$ then $\left(\overline{x_{i}}\land S\right)(\pmb a) = 1\cdot S(\pmb a) = 1$.\end{proof}

We find each $f(x_{i}\land S)$ with the blackbox. For each $i$, if $f(x_{i}\land S) = 1$ then set $a_{i}=1$, otherwise set $a_{i}=0$. Denote $y_{i} \sett \begin{cases}x_{i}&\text{ if }a_{i}=1\\\overline{x_{i}}&\text{ otherwise}\end{cases}$. $\pmb a$ satisfies every $y_{i}S$ by the above. We'll show that $\pmb a$ satisfies $S$. 
\begin{align*}\left(\bigwedge_{i=1}^{n}(y_{i}S)\right)(\pmb a) &= \left[\left(\bigwedge_{i=1}^{n}y_{i}\right) \land S\right](\pmb a)&[\because \alpha\land \alpha = \alpha]\\
\implies \prod_{i=1}^{n}(y_{i}\land S)(\pmb a) &= \left(\prod_{i=1}^{n}y_{i}(\pmb a)\right) \cdot S(\pmb a) &[\because(\alpha\land \beta)(\pmb a) = \alpha(\pmb a)\cdot\beta(\pmb a)]\\
\implies 1 &= \left(\prod_{i=1}^{n}y_{i}(\pmb a)\right) \cdot S(\pmb a) &[\because(y_{i}S)(a) = 1\forall i]\\
\implies 1 &=  S(\pmb a) &[\because y_{i}(\pmb a) = 1\forall i \text{ by construction}]
\end{align*}
\item 
\end{enumerate}


\newpage

\pb

Consider a family of decision problems indexed by a positive integer $k$:
\subsection*{RANK-$k$-SDP}
\textbf{Input}: Symmetric $N\times N$ matrices $A_{1},\cdots,A_{m}$ with entries in $\Q$, scalars $b_{1},\cdots,b_{m} \in\Q$. \\
\textbf{Question}: Is there a real symmetric matrix $X$ that satisfies the constraints
\begin{equation}\label{rk1}\Tr(A_{i}X) = b_{i},i \in[m], X \succeq 0, \rk(X) = k?\end{equation}

\soln

\subsubsection*{RANK-$1$-SDP:}
First notice that a symmetric psd matrix $X\in S^{n}$ has rank $1$ iff it has the form $xx^{\top}$ for some $x\in \R^{n}\smallsetminus\set{0}$. So in the underlying problem, the constraints can be rewritten as $b_{i}=\Tr(A_{i}X) = x^{\top}A_{i}x\forall i\in[m]$ and $x\ne 0$. Thus it is clear that \begin{equation}\label{ee}\exists X\in S^{n} \text{ s.t. } X\succeq0,\Tr(A_{i}X) = b_{i}\forall i \in[m], \rk(X) =1\iff \exists x\in \R^{n} \text{ s.t. } x^{\top}A_{i}x = b_{i}\forall i\in[m], x\neq0.\end{equation}
We will show a reduction \textbf{STABLE-SET} $\longrightarrow$ \textbf{RANK-$1$-SDP}. Let $(G,k)$ be an instance of \textbf{STABLE-SET} where graph $G$ has edges $E$ and vertices $[n]$. Recall that this is same as feasibility of some $v\in \R^{n}$ satisfying $v_{i}(1-v_{i}) = 0~\forall i\in [n], v_{i}v_{j} = 0 ~\forall \set{i,j}\in E, \sum\limits_{i\in[n]} v_{i} \ge k.$\\
Since each $v_{i}\in \set{0,1}, \sum_{i\in[n]} v_{i}$ whence the last constraint is equivalent to the existence of some $s\in \R$ such that $\left(\sum\limits_{i\in[n]} v_{i}\right)^{2} - k^{2}=s^{2}$. So \textbf{STABLE-SET} is the feasibility of some $v,s$ subject to \begin{equation}\tag{(Q)}\label{Q}\begin{aligned}
v\in\R^{n}, &~s\in\R\\
v_{i}(1-v_{i}) &= 0~\forall i\in [n]\\
v_{i}v_{j} &= 0 ~\forall \set{i,j}\in E\\
\left(\sum_{i\in[n]} v_{i}\right)^{2}&=k^{2}+s^{2}.
\end{aligned}\end{equation}

\begin{cl}
Feasibility of \ref{Q} is equivalent to the feasibility of $v,c,s$ subject to 
\begin{equation}\tag{(QQ)}\label{QQ}\begin{aligned}
v\in\R^{n}, c\in\R, &~s\in \R\\
c^{2}&=1\\
v_{i}(c-v_{i}) &= 0~\forall i\in [n]\\
v_{i}v_{j} &= 0 ~\forall \set{i,j}\in E\\
\left(\sum_{i\in[n]} v_{i}\right)^{2}-s^{2}&=k^{2}.
\end{aligned}\end{equation}
\end{cl}
\begin{proof}
If $(v,s)$. is feasible to \ref{Q}, then $(v,1,s)$ is clearly feasible to \ref{QQ}.\\
Now suppose $(v,c,s)$ is feasible to \ref{QQ}. So each $v_{i}\in \set{0,c}\forall i\in[n]$. This means $k^{2}+s^{2} = \left(\sum_{}{i} v_{i}\right)^{2} = \left(\sum_{\substack{i\in[n]\\v_{i}= c}} v_{i}\right)^{2} = c^{2}\left(\sum_{\substack{i\in[n]\\v_{i}\ne 0}} 1\right)^{2} = \left(\sum_{\substack{i\in[n]\\v_{i}\ne 0}} cv_{i}\right)^{2} = \left(\sum_{i} cv_{i}\right)^{2}$. This shows that $(cv,s)$ is feasible to \ref{QQ}.
\end{proof}

Notice how all constraints in \ref{QQ} are polynomial expressions with only constant terms and homogeneous quadratic terms. This perfectly matches with what we want in the original problem using \cref{ee}. To get an instance of \textbf{RANK-$1$-SDP}, take: \begin{itemize}
\item The size of matrices $N = n+2$,
\item $Q = e_{n+1}e_{n+1}^{\top}$ and $q=1$,
\item $B_{i} = E_{i,n+1} - 2 e_{i}e_{i}^{\top}$ and $r_{i}=0$ for each $i\in[n]$
\item $A_{ij} = E_{ij} = e_{i}e_{j}^{\top} + e_{j}e_{i}^{\top}$ (which is the $N\times N$ matrix with $1$ in positions $(i,j),(j,i)$ and $0$ elsewhere), $b_{ij}=b_{ji}=0$ for each edge $\set{i,j}\in E$,
\item $T = \sum\limits_{1\le i,j\le n}e_{i}e_{j}^{\top} - e_{N}e_{N}^{\top}, b = k^{2}$.
\end{itemize}The above are clearly all rational.

Now consider the constraints on $x$ \begin{equation}\tag{(HQ)}\label{HQ}\begin{aligned}x&\in\R^{N}\\
x^{\top}Qx&=q\\
x^{\top}B_{i}x &= r_{i}\forall i\in[n]\\
x^{\top}A_{ij}x &= b_{ij}\forall \set{i,j}\in E \\
x^{\top}Tx &= b\end{aligned}\end{equation}
We show that this answers the same question as \textbf{STABLE-SET} gives on $(G,k)$.
\begin{cl}
\ref{QQ} is feasible iff \ref{HQ} is feasible.
\end{cl}
\begin{proof}
A solution $(v,c,s)\in \R^{n}\times \R\times\R$ of \ref{QQ} corresponds to a solution  $x=(v,c,s)\in \R^{N}$ of \ref{HQ}. \begin{align*}
x^{\top}Qx &= c^{2}  & 1 &= q\\
x^{\top}B_{i}x &= 2v_{i}c-2v_{i}^{2} &0 &=r_{i}\\
x^{\top}A_{ij}x &= 2v_{i}v_{j} & 0&=b_{ij}\\
x^{\top}Tx &= (v_{1}+\cdots+v_{n})^{2} - s^{2} & k^{2} &= b.
\end{align*}
Left column matches the LHS and right column matches the RHS of each constraint in \ref{QQ} and \ref{HQ}.
\end{proof}
Notice that every solution $x$ of \ref{HQ} is automatically nonzero because of the first constraint: $x_{n+1}^{2}=1$. 

The above shows that $G$ has a stable set of size (at least) $k$ iff \ref{HQ} is feasible, which is a problem of \textbf{RANK-$1$-SDP}. This completes the proof that \textbf{RANK-$1$-SDP} is NP hard because we showed a reduction from an NP hard problem.

\subsubsection*{RANK-$k$-SDP:}
We will show a reduction \textbf{RANK-$1$-SDP} $\longrightarrow$ \textbf{RANK-$k$-SDP} for $k\ge 2$. Indeed, consider an instance of \textbf{RANK-$1$-SDP} with inputs which are $N\times N$ matrices $A_{1},\cdots,A_{m}\in \Q^{N\times N}$ and scalars $b_{1},\cdots,b_{m}\in\Q$. Consider the following constraints: 
\begin{equation}\label{rkk}
\begin{aligned}
X\in& S^{N+k-1}\\
\rk(X) &= k\\
X&\succeq 0\\
\Tr(E_{ij} X) =&~ 0 ~\forall N < i < N+k, 1\le j\le N\\
\Tr(E_{ij} X) =&~ 0 ~\forall N < i< j< N+k\\
\Tr(E_{ii} X) =&~ 0 ~\forall N < i < N+k\\
\Tr(\tilde A_{i}X) =&~ b_{i}~\forall i \in[m].
\end{aligned}
\end{equation}
Here $\tilde A_{i} = \begin{bmatrix}A_{i}&0\\0&0\end{bmatrix}\in S^{N+k-1}$ obtained by putting appropriate number of $0$'s.

We discuss the above linear constraints one by one. On slight inspection, the above linear constraints deal with four different blocks of $X = \begin{bmatrix}A & B\\B^{\top}&C\end{bmatrix}$ where $A\in S^{N}, B\in \R^{N\times(k-1)}, C\in S^{k-1}$.
\begin{itemize}
\item $\Tr(E_{ij} X) = 0 ~\forall N < i < N+k, 1\le j\le N:$ here $E_{ij} = e_{i}e_{j}^{\top}+e_{j}e_{i}^{\top}$. This is just saying that $X_{ij} = X_{ji} = 0$ whenever $i>N$ and $j\in [N]$. In other words $B=0$.
\item $\Tr(E_{ij} X) = 0 ~\forall N < i< j< N+k:$  This looks at the block $C$ because $j>i>N$. The constraint says that all off-diagonal entries of $C$ are $0$, that is, $C$ is a diagonal matrix.
\item $\Tr(E_{ii} X) = 0 ~\forall N < i < N+k:$ This says that the diagonal entries of $C$ are all $1$. Using the previous point, $C$ is the identity matrix of size $(k-1)\times (k-1)$.
\end{itemize}

So this makes $X$ look like $\begin{bmatrix}A&0\\0&I_{k-1}\end{bmatrix}$. Note that $\rk(X) = \rk(A) + \rk(I_{k-1}) = \rk(A) + k-1$, whence $\rk(X) = k\iff \rk(A)=1$. Further $\begin{bmatrix}A&0\\0&I_{k-1}\end{bmatrix}\succeq0\iff A\succeq 0$ because eigenvalues of a block matrix formed of two blocks stacked diagonally is the union of the eigenvalues of the two individual blocks.

\begin{cl}
\ref{rk1} is feasible iff \ref{rkk} is feasible.
\end{cl}
\begin{proof}
From the above bullet points and the rank discussion, it is clear that a rank $1$ solution $A \succeq 0$ of \ref{rk1} corresponds to a rank $k$ solution $\begin{bmatrix}A&0\\0&I_{k-1}\end{bmatrix}\succeq 0.$
\end{proof}

This completes the proof that \textbf{RANK-$k$-SDP} is NP hard because we showed a reduction from an NP hard problem.


\newpage

\pb

A polynomial $p(x)\sett p(x_{1},\cdots,x_{n})$ is nondecreasing with respect to a variable $x_{i}$ if $\frac{\partial p}{\partial x_{i}}(x)\ge 0\forall x\in\R^{n}$. Show that the problem of deciding whether a degree-$d$ polynomial with rational coefficients is nondecreasing with respect to a particular variable (e.g., $x_{1}$) is
\begin{enumerate}[label=(\roman*)]
\item in P if $d<5$.
\item NP-hard if $d\ge 5$.
\end{enumerate}

\soln

Set $d=\deg(p)$. Here $p\in\Q[x_{1},\cdots,x_{n}]$. For short, denote $\partial_{i}p \sett \frac{\partial p}{\partial x_{i}}$. 

First note that $\deg(\partial_{i}(p)(x)) < d$ for each $i\in[n]$. This is because (fixing some $i\in[n]$) if $p(x) = x_{i}^{k}q_{k}(x_{-i}) + \cdots + x_{i} q_{1}(x_{-i}) + q_{0}(x_{-i})$ with all $q_{j}\in \Q[x_{-i}]$ ($x_{-i}$ denotes the vector with all variables in $x$ without $x_{i}$), then $d = \max\limits_{0\le j\le k}\set{j+\deg(q_{j})}$ whence $\deg (\partial_{i}p) = \max\limits_{1\le j\le k}\set{j-1+\deg(q_{j})} \le \max\limits_{0\le j\le k}\set{j-1+\deg(q_{j})} = d-1$ as a polynomial in $\Q[x]$.

The problem as mentioned, for each degree $d$, takes input $n$, the number of variables, the rational coefficients that make a polynomial, and an index $i$; and answers the question if the polynomial is non-decreasing in variable $x_{i}$. Denote the input size by $N$ - that is the number of coefficients till degree $d$. The degree $d$ of this polynomial can in fact be found in polynomial time in $N$. So there is a problem for each degree. Call it \textbf{MONO-d}.

\begin{enumerate}[label=(\roman*)]
\item We will show this by cases on degree of $d'=\deg(\partial_{i}p(x))$. Note that $d'\le 4$ if $d\le 5$.
\begin{itemize}
\item[$d=0:$] Then $\partial_{i}p = 0\ge 0$. So non-increasing. Thus we answered in constant time.
\item[$d'=0:$] Then $\exists b\in \Q$ such that $\partial_{i}p(x) = b$. Non-negativity of the rational constant $b$ can be checked in constant time.
\item[$d'=1:$] The required derivative looks like $\partial_{i}p(x) = b^{\top}x + c$ for some $b\in\Q^{n}, c\in\Q$. This is $\ge 0$ everywhere on $\R^{n}$ iff $b\ne 0$ and $c\ge 0$. Nonzero-ness of $b$ can be checked in linear (in $n$) time and $c\ge 0$ can be checked in constant time.
\item[$d'=2$:] The required derivative looks like $\partial_{i}p(x) = \frac12x^{\top}Ax - b^{\top}x + c$ for some $A\in\Q^{n\times n},b\in\Q^{n}, c\in\Q$ and $A$ symmetric. Then:
\begin{itemize}
\item Say $A \not\succeq 0$. This is checked in $\mathcal{O}(n^{3})$ time. Then the function $\partial_{i}p$ is unbounded below: along the direction of some eigenvector with negative eigenvalue.
\item So $A\succeq 0$ now. We have reduced to the problem of minimizing the convex function $\partial_{i}p(x)$ where $x\in\R^{n}$. Recall that if $x\in\R^{n}$ is a minima of this function then it's a critical point. Conversely if $x\in \R^{n}$ is a critical point, then $\nabla(\partial_{i}p)(x) = 0\implies \nabla(\partial_{i}p(x))(y-x)\ge0\forall y\in\R^{n}$ whence it is a minima (convexity was needed here). If there is no $x\in\R^{n}$ such that $(Ax-b=)\partial_{i}p(x)=0$, then there is no minima and the problem is unbounded below. Otherwise assume there is a critical point $v \in\R^{n}$, that is, $Av=b$. In other words, $b$ is in the column span ($=$ row transpose span because $A$ is symmetric) of $A$. Then we claim every critical point takes the same objective. Say $u$ is another critical point. Clearly $Av = Au = b$ whence $A^{\top}(u-v)=A(u-v)=0$. Since $b$ is in the column span of $A$, $b^{\top}(u-v)=0$. But $\partial_{i}p(v) = -\frac12b^{\top}v+c = \frac12b^{\top}u+c=p(u)$. It follows that any critical point $v$ gives a global minima with value $-\frac12b^{\top}v+c$. 
\end{itemize} 
Thus our polynomial time algorithm is:
\begin{enumerate}
\item Differentiate $p(x)$ to get $A,b,c$ such that $\partial_{i}p(x) = \frac12x^{\top}Ax-b^{\top}x+c$.
\item Check if $A\succeq 0$. If not, conclude that $p$ is not non-decreasing wrt $x_{i}$ and {\tt{EXIT}} (because $\partial_{i}p$ is then unbounded below). Otherwise go to the next step.
\item Check if $b$ is in the column span of $A$. If not, conclude that $p$ is not non-decreasing wrt $x_{i}$ and {\tt{EXIT}} (because $\partial_{i}p$ is then unbounded below). Otherwise go to the next step.
\item Otherwise $\set{x\in\R^{n}\st Ax=b}\neq\varnothing$. Find a solution $\overline x$ to $Ax=b$. The objective $\partial_{i}p(\overline x)$ becomes $-\frac{1}{2}b^{\top}\overline x+c$. If this value is $\ge 0$, conclude that $p$ is non-decreasing wrt $x_{i}$ and {\tt{EXIT}}. Otherwise conclude that $p$ is not non-decreasing wrt $x_{i}$ and {\tt{EXIT}}.
\end{enumerate}
This is correct because of the above discussion. This runs in polynomial time in size of input because step (a) takes same number of steps as number of coefficients; step (b) takes $\mathcal{O}(n^{3}) \le \mathcal{O}(N^{3})$ steps; step (c) can be done in polynomial time (in $n$) again (say Gaussian elimination); and step (d) can again be done in polynomial steps in $n$. So overall, the number of steps this algorithm takes is polynomial in the input size $N$.
\item[$d'=3:$] So $\partial_{i}p(x)$ restricted to $\set{x\in\R^{n}\st x_{1}=x_{2}=\cdots=x_{n}}$ gives a $3-$degree polynomial in one variable which, we know from elementary theory, is unbounded below.
\end{itemize}
\item We'll now show a reduction \textbf{COPOS} $\longrightarrow$ \textbf{MONO-$d$} for $d\ge 5$, where \textbf{COPOS} takes input a matrix $M$ and decides whether it is copositive. Given an input $M\in\Q^{n\times n}$ to \textbf{COPOS}, we'll give the input $\ds p(x_{1},\cdots,x_{n+1})\sett x_{n+1} \left(\sum_{1\le i\le j\le n}M_{ij}x_{i}^{2}x_{j}^{2}\right) + x_{1}^{d}$ and variable index $n+1$ to \textbf{MONO-$d$}. This makes $\deg$ of the first term to be $5$ which is why we want $d\ge 5$. Then the following is immediate.
\begin{cl}
$M$ is copositive iff $q$ is non-decreasing wrt $x_{n+1}$.
\end{cl}
\begin{proof}
Note that $\ds\partial_{n+1}p(x) = \sum_{1\le i\le j\le n}M_{ij}x_{i}^{2}x_{j}^{2} = \begin{bmatrix}x_{1}^{2}\\\vdots\\x_{n}^{2}\end{bmatrix}^{\top}M\begin{bmatrix}x_{1}^{2}\\\vdots\\x_{n}^{2}\end{bmatrix}\ge 0$ for all $x\in\R^{n+1}$ iff $v^{\top}Mv\ge 0\forall v\in\R^{n} $ such that $v\ge 0$ iff $M$ is copositive.
\end{proof}
Since \textbf{COPOS} is known to be NP-hard, the above reduction shows that \textbf{MONO-$d$} is NP-hard.
\end{enumerate}

\newpage
\pb
\begin{enumerate}[leftmargin=*]
\item In the file \texttt{regression\textunderscore data.mat}, you are given $20$ points $(x_{i}, f_{i})$ in $\R^{2}$ where $(x_{i})_{i=1,\cdots,20}$ are the entries of the vector \texttt{xvec} and $(f_{i})_{i=1,\cdots,20}$ are the entries of the vector \texttt{fvec}. The goal is to fit a polynomial of degree $7$ \begin{equation}\label{pol} p(x) = c_{0} +c_{1}x+\cdots+c_{7}x^{7}\end{equation} to the data to minimize least square error: \begin{equation}\label{ls}\min_{c_{1},\cdots,c_{7}} \sum_{i=1}^{20}(f_{i}-p(x_{i}))^{2}.\end{equation} The data comes from noisy measurements of an unknown function that is a priori known to be nondecreasing (e.g., the number of calories you intake as a function of the number of Big Macs you eat).
\begin{enumerate}[label=(\alph*)]
\item If the underlying function is truly monotone and the noise is not too large, one may hope that least squares would automatically respect the monotonicity constraint. Solve (\ref{ls}) to see if this is the case. Plot the optimal polynomial you get and report the optimal value.
\item Resolve (\ref{ls}) subject to the constraint that the polynomial (\ref{pol}) is nondecreasing. Plot the optimal polynomial you get and report the optimal value.
\end{enumerate}
\end{enumerate}



%{\includepdf[pages=-,pagecommand={\label{pdf:code}}]{graph/graph.pdf}}


\end{document}

