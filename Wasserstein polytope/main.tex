% !TEX TS-program = pdflatexmk
\documentclass[12pt]{amsart}
\usepackage[letterpaper, portrait, left = 1in, right = 1in, top = 1.2in, bottom=1.5in]{geometry} 
%\usepackage{setspace} \doublespacing
%\usepackage[letterpaper, portrait, margin=1.3in]{geometry}
\usepackage[table,xcdraw]{xcolor}
\usepackage{amssymb}
\usepackage{amsfonts}
\usepackage{longtable}
\usepackage{amsmath,amsthm}
\usepackage{enumitem}
\usepackage[utf8]{inputenc}
\usepackage{mathtools}
\usepackage{graphicx}
\usepackage{parskip}
\usepackage{multicol}
\usepackage{listings}
\usepackage[skip=0.25pt]{caption}
\usepackage[mathscr]{euscript}
\usepackage{quiver}
\setlength{\parindent}{0pt}
\usepackage{thm-restate}
\definecolor{vividburgundy}{rgb}{0.62, 0.11, 0.21}
\usepackage[driverfallback=hypertex,pagebackref=false,colorlinks,citecolor=vividburgundy]{hyperref}
\usepackage[capitalize]{cleveref}
%\usepackage[cmintegrals,cmbraces]{newtxmath}
%\usepackage{ebgaramond-maths}

%\usepackage{fourier}
%----------FONT OPTIONS----------
% sans-serif
%\usepackage[sfdefault]{FiraSans}
 %\usepackage[sfdefault]{roboto}
% \usepackage[sfdefault]{noto-sans}
%\usepackage[default]{sourcesanspro}

% serif
%\usepackage{CormorantGaramond}

%\usepackage{charter}
\usepackage[T1]{fontenc}
\usepackage{cleveref}
\definecolor{dg}{RGB}{10, 100, 10}
\setlength{\parindent}{0in}
\renewcommand{\qed}{$\hfill\blacksquare$}
% \newtheoremstyle{style}{2pt}{1pt}{\normalfont}{}{\bfseries}{\\}{0cm}{}
% \theoremstyle{style}
\newtheorem{lemma}{Lemma}[section]
% \newtheorem{lemma}{Lemma}
\newtheorem{thm}[lemma]{Theorem}
\newtheorem{prop}[lemma]{Proposition}
\newtheorem{cor}[lemma]{Corollary}
\newtheorem{conj}[lemma]{Conjecture}
\newtheorem{cl}[lemma]{Claim}
\newtheorem{rmk}{Remark}
\newtheorem{defn}[lemma]{Definition}
\newtheorem{qs}{Question}
\newtheoremstyle{styleS}{}{}{\color{dg}}{}{\color{dg}\bfseries}{. }{0cm}{}
\theoremstyle{styleS}
\newtheorem*{sol}{Solution}
\newtheoremstyle{style1}{}{}{\normalfont}{}{\bfseries}{. }{0cm}{}
\theoremstyle{style1}
\newtheorem{prob}{Problem}[section]
\newtheorem*{prb}{Problem}
\newtheoremstyle{style2}{1pt}{4pt}{\normalfont}{}{\itshape}{. }{0cm}{}
\theoremstyle{style2}
\newtheorem{ex}[lemma]{Example}
\newtheorem*{pf}{Proof}

\newcommand{\norm}[2]{
\left\lVert #1 \right\rVert_{#2}
}
\usepackage{mathrsfs}
%\usepackage[table,xcdraw]{xcolor}
\usepackage{booktabs}
\usepackage{tikz}
\usetikzlibrary{matrix}
\renewcommand{\l}{\ell}


\newcommand{\fA}{{\mathfrak{A}}}   \newcommand{\fB}{{\mathfrak{B}}}
\newcommand{\fC}{{\mathfrak{C}}}   \newcommand{\fD}{{\mathfrak{D}}}
\newcommand{\fE}{{\mathfrak{E}}}   \newcommand{\fF}{{\mathfrak{F}}}
\newcommand{\fG}{{\mathfrak{G}}}   \newcommand{\fH}{{\mathfrak{H}}}
\newcommand{\fI}{{\mathfrak{I}}}   \newcommand{\fJ}{{\mathfrak{J}}}
\newcommand{\fK}{{\mathfrak{K}}}   \newcommand{\fL}{{\mathfrak{L}}}
\newcommand{\fM}{{\mathfrak{M}}}   \newcommand{\fN}{{\mathfrak{N}}}
\newcommand{\fO}{{\mathfrak{O}}}   \newcommand{\fP}{{\mathfrak{P}}}
\newcommand{\fQ}{{\mathfrak{Q}}}   \newcommand{\fR}{{\mathfrak{R}}}
\newcommand{\fS}{{\mathfrak{S}}}   \newcommand{\fT}{{\mathfrak{T}}}
\newcommand{\fU}{{\mathfrak{U}}}   \newcommand{\fV}{{\mathfrak{V}}}
\newcommand{\fW}{{\mathfrak{W}}}   \newcommand{\fX}{{\mathfrak{X}}}
\newcommand{\fY}{{\mathfrak{Y}}}   \newcommand{\fZ}{{\mathfrak{Z}}}

\newcommand{\cA}{{\mathcal{A}}}   \newcommand{\cB}{{\mathcal{B}}}
\newcommand{\cC}{{\mathcal{C}}}   \newcommand{\cD}{{\mathcal{D}}}
\newcommand{\cE}{{\mathcal{E}}}   \newcommand{\cF}{{\mathcal{F}}}
\newcommand{\cG}{{\mathcal{G}}}   \newcommand{\cH}{{\mathcal{H}}}
\newcommand{\cI}{{\mathcal{I}}}   \newcommand{\cJ}{{\mathcal{J}}}
\newcommand{\cK}{{\mathcal{K}}}   \newcommand{\cL}{{\mathcal{L}}}
\newcommand{\cM}{{\mathcal{M}}}   \newcommand{\cN}{{\mathcal{N}}}
\newcommand{\cO}{{\mathcal{O}}}   \newcommand{\cP}{{\mathcal{P}}}
\newcommand{\cQ}{{\mathcal{Q}}}   \newcommand{\cR}{{\mathcal{R}}}
\newcommand{\cS}{{\mathcal{S}}}   \newcommand{\cT}{{\mathcal{T}}}
\newcommand{\cU}{{\mathcal{U}}}   \newcommand{\cV}{{\mathcal{V}}}
\newcommand{\cW}{{\mathcal{W}}}   \newcommand{\cX}{{\mathcal{X}}}
\newcommand{\cY}{{\mathcal{Y}}}   \newcommand{\cZ}{{\mathcal{Z}}}

\newcommand{\sA}{{\mathscr{A}}}   \newcommand{\sB}{{\mathscr{B}}}
\newcommand{\sC}{{\mathscr{C}}}   \newcommand{\sD}{{\mathscr{D}}}
\newcommand{\sE}{{\mathscr{E}}}   \newcommand{\sF}{{\mathscr{F}}}
\newcommand{\sG}{{\mathscr{G}}}   \newcommand{\sH}{{\mathscr{H}}}
\newcommand{\sI}{{\mathscr{I}}}   \newcommand{\sJ}{{\mathscr{J}}}
\newcommand{\sK}{{\mathscr{K}}}   \newcommand{\sL}{{\mathscr{L}}}
\newcommand{\sM}{{\mathscr{M}}}   \newcommand{\sN}{{\mathscr{N}}}
\newcommand{\sO}{{\mathscr{O}}}   \newcommand{\sP}{{\mathscr{P}}}
\newcommand{\sQ}{{\mathscr{Q}}}   \newcommand{\sR}{{\mathscr{R}}}
\newcommand{\sS}{{\mathscr{S}}}   \newcommand{\sT}{{\mathscr{T}}}
\newcommand{\sU}{{\mathscr{U}}}   \newcommand{\sV}{{\mathscr{V}}}
\newcommand{\sW}{{\mathscr{W}}}   \newcommand{\sX}{{\mathscr{X}}}
\newcommand{\sY}{{\mathscr{Y}}}   \newcommand{\sZ}{{\mathscr{Z}}}

\newcommand{\ta}{{\tilde{a}}}   \newcommand{\tb}{{\tilde{b}}}
\newcommand{\tc}{{\tilde{c}}}   \newcommand{\td}{{\tilde{d}}}
\newcommand{\te}{{\tilde{e}}}   \newcommand{\tf}{{\tilde{f}}}
\newcommand{\tg}{{\tilde{g}}}   
\newcommand{\ti}{{\tilde{i}}}   \newcommand{\tj}{{\tilde{j}}}
\newcommand{\tk}{{\tilde{k}}}   \newcommand{\tl}{{\tilde{l}}}
\newcommand{\tm}{{\tilde{m}}}   \newcommand{\tn}{{\tilde{n}}}
		         	\newcommand{\tp}{{\tilde{p}}}
\newcommand{\tq}{{\tilde{q}}}   \newcommand{\tr}{{\tilde{r}}}
\newcommand{\ts}{{\tilde{s}}}   
\newcommand{\tu}{{\tilde{u}}}   \newcommand{\tv}{{\tilde{v}}}
\newcommand{\tw}{{\tilde{w}}}   \newcommand{\tx}{{\tilde{x}}}
\newcommand{\ty}{{\tilde{y}}}   \newcommand{\tz}{{\tilde{z}}}

\newcommand{\red}{{\color{red}red}}
\newcommand{\blue}{{\color{blue}blue}}

\newcommand{\into}{\hookrightarrow}
\newcommand{\onto}{\twoheadrightarrow}
\newcommand\N{\ensuremath{\mathbb{N}}}

%\newcommand\L{\ensuremath{\mathbb{L}}}
\newcommand{\bP}{\mathbb{P}}
\newcommand\M{\ensuremath{\mathbb{M}}}
\newcommand\R{\ensuremath{\mathbb{R}}}
\newcommand\Z{\ensuremath{\mathbb{Z}}}
\renewcommand\O{\ensuremath{\emptyset}}
\newcommand\Q{\ensuremath{\mathbb{Q}}}
\newcommand\C{\ensuremath{\mathbb{C}}}
\newcommand{\K}{\ensuremath{\mathbb{K}}}
\newcommand\F{\ensuremath{\mathbb{F}}}
\newcommand{\aff}{\ensuremath{\mathbb{A}}}
\newcommand{\proj}{\ensuremath{\mathbb{P}}}
\newcommand{\dd}{\mathrm{d}}
\newcommand{\m}{\ensuremath{\mathfrak{m}}}
\newcommand{\p}{\ensuremath{\mathfrak{p}}}
\newcommand{\n}{\ensuremath{\mathfrak{n}}}
\renewcommand{\phi}{\varphi}
\renewcommand{\qedsymbol}{\ensuremath{\blacksquare}}
%\newcommand{\st}{\;|\;}
\newcommand{\st}{%
  \nonscript\;
  \ifnum\currentgrouptype=16
    \;\middle|\;
  \else
    \;|\;
  \fi
  \nonscript\;}
\newcommand{\ltr}{\par \noindent \framebox[1\width]{ $\implies$ } \hspace{.2cm}}
\newcommand{\rtl}{\par \noindent \framebox[1\width]{ $\impliedby$ } \hspace{.2cm} }
\newcommand{\abs}[1]{\left| #1 \right|}
\newcommand{\inner}[2]{\left\langle #1, #2 \right\rangle}
\newcommand{\E}[1]{\mathbb E\left[ #1 \right]}
\newcommand{\e}[1]{\exp\left( #1 \right)}
\renewcommand{\P}[1]{\mathbb P\left[ #1 \right]}
\newcommand{\Var}[1]{\text{Var}\left[ #1 \right]}
\newcommand*\circled[1]{\tikz[baseline=(char.base)]{
            \node[shape=circle,draw,inner sep=2pt] (char) {#1};}}
\newcommand{\ds}{\displaystyle}

\DeclareMathOperator{\sym}{Sym}
\DeclareMathOperator{\mds}{MDS}
\DeclareMathOperator{\Tor}{Tor}
\DeclareMathOperator{\Ext}{Ext}
\DeclareMathOperator{\adj}{adj}
\DeclareMathOperator{\Tr}{Tr}
\DeclareMathOperator{\GL}{GL}
%\DeclareMathOperator{\Tr}{Tr}
\DeclareMathOperator{\orbit}{Or}
\DeclareMathOperator{\stab}{Stab}
\DeclareMathOperator{\fix}{Fix}
\DeclareMathOperator{\re}{Re}
\DeclareMathOperator{\im}{Im}
\DeclareMathOperator{\ord}{Ord}
\DeclareMathOperator{\mspec}{mSpec}
\DeclareMathOperator{\spec}{Spec}
\DeclareMathOperator{\frob}{Frob}
\DeclareMathOperator{\id}{Id}
\DeclareMathOperator{\colim}{colim}
\DeclareMathOperator{\loc}{loc}
\DeclareMathOperator{\res}{Res}
\DeclareMathOperator{\rad}{rad}
\DeclareMathOperator{\Res}{Res}
\DeclareMathOperator{\diam}{diam}
\DeclareMathOperator{\arcsec}{arcsec}
\DeclareMathOperator{\arccot}{arccot}
\DeclareMathOperator{\len}{len}
\DeclareMathOperator{\area}{area}
\DeclareMathOperator{\vol}{vol}
\DeclareMathOperator{\ev}{ev}
\DeclareMathOperator{\sgn}{sgn}
\DeclareMathOperator{\supp}{supp}
\DeclareMathOperator{\diff}{d}
\DeclareMathOperator{\Dom}{Dom}
\DeclareMathOperator{\rk}{rank}
\renewcommand{\d}{\diff}
\let\oldend\endlinechar
\renewcommand{\endlinechar}{\oldend}
\newcommand{\open}{\underset{\text{open}}{\subset}}
\newcommand{\divides}{\mathbin{|}}
\newcommand{\set}[1]{\ensuremath{\left\{#1\right\}}}
\newcommand{\sett}{\coloneqq}
\newcommand*\isomap{%
  \xrightarrow{\raisebox{-0.9ex}[0ex][0ex]{$\sim$}}%
}
\renewcommand{\epsilon}{\varepsilon}
\newcommand{\fa}{~\forall~}
%\usepackage[nobottomtitles*]{titlesec}
\usepackage{titletoc}
%\titleformat{\section}[runin]
%{\normalfont\Large\bfseries}
%{}{0pt}{}%
%[\ifthenelse{\equal{\thesection}{0}}{\\\vspace*{0pt}}{\space\thesection}]
\newcommand{\sint}{\sin\theta}
\newcommand{\cost}{\cos\theta}
\newcommand{\tant}{\tan\theta}
\newcommand{\lb}{\left[}
\newcommand{\rb}{\right]}
\newcommand{\lp}{\left(}
\newcommand{\rp}{\right)}
\newcommand{\br}[1]{\lb#1\rb}
\newcommand{\pa}[1]{\lp#1\rp}
\usepackage{pdfpages}
\usepackage{fancyhdr}
	\pagestyle{fancyplain}
	\fancyhf{}
	\fancyhead[C]{\thepage}


\usepackage[
backend=biber,
style=alphabetic,giveninits,
citestyle=ieee-alphabetic,
natbib=true,
uniquelist=false,
maxnames=10,
sorting=ynt
]{biblatex}
%
\addbibresource{ref.bib}

\title{Characterizing vertices of waasserstein ball}
\author{s}
\date{July 2024}


\begin{document}
\begin{abstract}
We study the combinatorics of the Wasserstein$-1$ metric for various distances.
\end{abstract}
\maketitle

\section{Introduction}
The probability simplex 
$$ \Delta_{n-1}  \sett \set{(p_1, \ldots, p_n) \st \sum_{i=1}^n p_i = 1 \text{ and } p_i \geq 0 \fa i=1, \ldots,n }$$
consists of probability distributions of a discrete random variable with a state space of size $n$. We  take this state space to be $[n] := \{1, \ldots, n\}$. A {\it statistical model} $\cM$ is a subset of $\Delta_{n-1}$ which represents distributions to which a hypothesized unknown distribution $\pmb\nu$ belongs. Typically, after collecting data $\pmb u=(u_1, \cdots, u_n)$ where $u_i$ is the number of times outcome $i$ is observed, one forms 
the empirical distribution $\bar{\pmb \mu} = \frac{1}{N} \pmb u$  where $ N = \sum\limits_{i=1}^n u_i$ is the sample size. Note that $\bar{\pmb \mu} \in \Delta_{n-1}$. To estimate the unknown distribution $\pmb \nu$, a standard approach is to locate $\pmb \nu \in \cM$, that is a ``closest" point to $\bar {\pmb\mu}$. For instance, $\pmb\nu$ can be taken to be the maximum likelihood estimator \cite[Chapter 7]{sullivant2018algstat} of $\bar{\pmb \mu}$. In this case, $\pmb \nu$ is the point on $\cM$ that minimizes the Kullback-Leibler divergence from $\bar{\pmb \mu}$ to $\cM$. However, Kullback-Leibler divergence is not a metric, and the maximum likelihood estimator does not minimize a true distance function from $\bar{\pmb\mu}$ to $\cM$.

For the above density estimation problem, one can use a distance minimization approach if the state space $[n]$ is also a metric space. A metric on $[n]$ is a collection of nonnegative real numbers $d_{ij}$ for $i,j \in [n]$ such that $d_{ii} = 0$ for all $i \in [n]$, $d_{ij} = d_{ji}$, and the triangle inequality $d_{ik} \leq d_{ij} + d_{jk}$ holds for all $i,j,k \in [n]$. Sometimes, the metric on $[n]$ is written as an $n\times n$ nonnegative symmetric matrix $d=(d_{ij})_{i,j\in [n]}$. Common examples include the discrete metric (all $d_{ij}=1$), the $L_1$ metric ($d_{ij} = \abs{i-j}$), the $L_0$ metric, and the Hamming distance metric.

For two probability distributions $\pmb \mu$ and $\pmb \nu$ in
$\Delta_{n-1}$, the optimal value $W_d(\pmb \mu,\pmb \nu)$ of the following linear program is the {\it Wasserstein distance} between $\pmb\mu$ and $\pmb\nu$ based on the metric $(d_{ij})$:
\begin{equation} \label{eq:Wasserstein program}
\mbox{maximize} \quad \sum_{i=1}^n (\mu_i - \nu_i)x_i \quad \mbox{subject to} \quad |x_i -
x_j| \leq d_{ij} \,\, \mbox{for all} \,\, 1 \leq i < j \leq n.
\end{equation}
This means we can define $W_{d}(\pmb\mu,\pmb\nu)$ for any pair of vectors $\pmb\mu,\pmb\nu$ satisfying $\pmb 1^{\top}\pmb\mu=\pmb1^{\top}\pmb\nu$.
One should note that the constraint set of the variable $\pmb x$ in problem \ref{eq:Wasserstein program} is unbounded and that if $\pmb \alpha\in H_{n-1}\sett\set{\pmb x\in\R^{n}\st \pmb 1^{\top}\pmb x=0}$ and $\lambda\in\R$ then $\pmb \alpha^\top(\pmb x+\lambda\pmb 1) = \pmb \alpha^\top\pmb x$. So we can equivalently formulate it as $$W_{d}(\pmb \mu,\pmb\nu) = \max\set{(\pmb \mu - \pmb\nu)^\top\pmb x\st \pmb x\in H_{n-1}, \abs{x_i-x_j}\le d_{ij}\fa i,j}$$ which has a bounded constraint set. The constraint set of this linear program is called the \textit{Lipshitz polytope} $$ P_d = \set{ \pmb x \in H_{n-1} \st \abs{x_i - x_j} \leq d_{ij} \fa~ 1 \leq i < j \leq n } .$$ 


The Wasserstein distance $W_d(\pmb\mu,\cM)$ from $\pmb\mu \in \Delta_{n-1}$ to a set $\pmb\cM$ is the infimum of $W_d(\pmb\mu,\pmb\nu)$ as $\pmb\nu$ ranges over $\cM$: 
\begin{equation} \label{eq: Wasserstein to model}
W_d(\pmb\mu, \cM) := \min_{\pmb\nu\in \cM} W_d(\pmb\mu, \pmb\nu).  
\end{equation}
This has been successfully used to construct a version of Generative Adversarial Networks \cite{WGAN17} where $W_d(\cdot, \cM)$ is used as the loss function. However, for large $n$, computing $W_d(\pmb\mu, \cM)$ exactly is not feasible with the current state of knowledge. If we take $\cM=\set{\pmb\nu}$ we recover the original Wasserstein distance $W_d(\pmb \mu,\pmb \nu) = \min \set{\lambda \geq 0 \st \pmb \nu \in \pmb \mu + \lambda B }.$

In this paper our starting point is \cite{Wasserstein2,ccelik2021wasserstein} to study the combinatorics of the Wasserstein unit ball. Such combinatorics is governs the combinatorial complexity (contrast against algebraic complexity) of problem \ref{eq: Wasserstein to model}. We first recall this approach. 

The Wasserstein distance $W_d$ induced  by the finite metric $d$ on $[n]$ defines a 
norm on $H_{n-1}$ namely $$\norm{\pmb\alpha}d = \norm{\pmb\alpha}d^{W} = \max\set{\pmb \alpha^{\top}\pmb\mu \st \pmb x\in H_{n-1}, \abs{x_{i}-x_{j}}\le d_{ij}\fa~1\le i<j\le n}.$$ The unit ball of this norm is the polytope
\begin{equation}\label{eq:conv}B_{d} = \mathrm{conv} \set{ \frac{1}{d_{ij}}(\pmb e_i - \pmb e_j) \, : \, 1 \leq i < j \leq n },\end{equation}
where $B$ lies in the hyperplane $H_{n-1}$ and is the dual of the {\it Lipshitz polytope} $P_{d}$. It is well known that the $k$ dimensional facets of $P_{d}$ are in on-to-one correspondence with the $k$ codimensional facets of $B_{d}$. In other words, the number of $k$ dimensional facets of $P_{d}$ is equal to the number of $n-2-k$ dimensional facets of $B_{d}$.

\section{Vertices of $B_{d}$ with $d$ induced by a graph}

Consider the discrete metric $d$ on $[n]$. Formally this is given by $d_{ij} = 1\fa~i\ne j$. \cite{rootpoly, ccelik2021wasserstein} prove that the number of $k$ dimensional facets of $B_{d}$ is $\ds{n\choose k+2}(2^{k+2}-2)$. In particular, the number of vertices ($k=0$) is $n(n-1)$. This is the maximum number of possible vertices a Wasserstein ball can have, for any metric $d$, by the description in \Cref{eq:conv}. Here is an alternate way to think about the metric $d$. Consider the complete graph $K_{n}$ on $n$ vertices, labelled with $[n]$, so every vertex is connected to every other vertex by an edge. Then $d_{ij}=1$ is the length of the shortest path to reach $j$ from $i$ on this graph. This graph has precisely $\ds n\choose 2$ edges. Soon it will turn out that the number of vertices of $B_{d}$ being double the number of edges is not a coincidence. Further, based on this example, we propose the following definition.

\begin{defn}[Wasserstein metric based on a graph]
Let $G = ([n],E,w)$ be a connected weighted undirected graph without self loops that has vertices $[n]$, edges $E$ and non-negative weights given by $w:E^{2}\to\R_{\ge 0}$. If $G$ is unweighted, we simply treat $G$ as a weighted graph with weights of all edges as $1$. Define $d_{ij}$ to be the weighted length of the shortest path from vertex $i$ to $j$. The \textup{Wasserstein metric} $W_{G}$ \textup{based on graph} $G$ is defined to be the Wasserstein metric $W_{d}$ based on $d$. 
\end{defn}

Corresponding to the abovementioned Wasserstein metric $W_{G}$, its unit ball in $H_{n-1}$ will be denoted by $B_{G}$.

\begin{ex}
The metric induced by an unweighted line graph on $n$ vertices is said to be the $L_{1}$ metric on $[n]$. Let's look at $n=3$. So $G$ is $1\text{\textemdash}2\text{\textemdash}3$. The corresponding metric is given by $d_{ij}=\abs{i-j}$. According to \Cref{eq:conv}, $B_{G}$ is the convex hull of the points $\ds \pmb u_{\pm} = \pm(1,-1,0), \pmb v_{\pm}=\pm (0,1,-1), \pmb w_{\pm}=\pm(0.5,0,-0.5)$. But $\pmb w_{\pm} = \frac12 \pmb u_{\pm} +\frac12\pmb v_{\pm}$ hence not vertices. The vertices of $B_{G}$ turn out to be exactly $\pmb u_{\pm},\pmb v_{\pm}$; so total $4$ in number. Again observe that the number of vertices of $B_{G}$ is double the number of edges in $G$.
\end{ex}

Next we will turn towards the key result in this section, namely the phenomenon we observed both for the discrete and $L_{1}$ metric. Such results have been studied for weighted graphs in \cite[Theorem 2, \S 3.1]{finitemetric}, however our proof technique is purely combinatorial and constructions are slightly different.

\begin{thm}
Let $G=([n],E)$ be a connect unweighted undirected graph without self loops on $n$ vertices. Then the unit ball $B_{G}$ of the Wasserstein metric induced by $G$ has precisely $2\abs E$ vertices, namely $\set{\pmb e_{i}-\pmb e_{j}\st \set{i,j}\in E}$.
\end{thm}

Before starting the proof right away, we present an observation that was key in the examples of discrete and $L_{1}$ metrics. Our graph $G$ is connected, unweighted and undirected. If shortest path from $i$ to $j$ is $i = x_{1} \to x_{2}\to\cdots \to x_{p}=j$ then $d_{ij} = p-1$  and $\ds\frac{\pmb e_{j}-\pmb e_{i}}{d_{ij}} = \frac{\pmb e_{j}-\pmb e_{i}}{p-1} = \frac{1}{p-1} \sum\limits_{t=1}^{p-1}(\pmb e_{t+1}-\pmb e_{t}) = \frac{1}{p-1} \sum\limits_{t=1}^{p-1} \frac{\pmb e_{x_{t+1}}-\pmb e_{x_{t}}}{d_{x_{t}x_{t+1}}}$. In other words, $\ds\frac{\pmb e_{j}-\pmb e_{i}}{d_{ij}}$ is never a vertex of $B_{G}$ because it is a convex combination of some other points in $B_{G}$ corresponding to edges in $G$.

{\color{red}If we want to determine a $d$, for given $n$ and number of vertices $2\alpha$, for which the constraint matrix $M$ satisfies that its rank is $2\alpha$, we want to find a rank $2$ matrix $M$ with the rows being $\frac{\pmb e_{i}-\pmb e_{j}}{d_{ij}}$, such that its rank is $2\alpha$, then equivalently we want to search for a matrix $X=M^{\top}M\succeq 0$ with rank $2\alpha$.}

\newpage
%\nocite{*}
\printbibliography

\end{document}
