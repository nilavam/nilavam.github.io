\begin{sol}
The given statement is equivalent to showing the existence of an invertible $B$ such that $A^t=B^{-1}AB$. This is just saying that $A,A^t$ are similar. Since we are working over $\C$, we can simply work with JCF. This suffices because if $A=X^{-1}JX$ where $J$ is the JCF of $A$, then $A^{t}=B^{-1}AB$ is equivalent to saying that $YJ^tY^{-1} = B^{-1}X^{-1}JXB$ where $Y = X^t$, which is equivalent to saying that $J^t = (XBY)^{-1}X (XBY)$. This is simply saying that $J$ is similar to its transpose. Since $J$ is made of block matrices, and transpose treats every square block independently, it is enough to show that every Jordan block is similar to its transpose.

To see this, we start with a Jordan block $J$ of size $n\times n$ and eigenvalue $\lambda$. In the standard basis $\pmb e=(e_1,\cdots,e_n)$, the action of $J$ is given by $Je_1 = \lambda e_1$ and $Je_j = \lambda e_j+e_{j-1}$ for $1<j\leq n$. Now we look at the matrix in the basis $\pmb f = (f_1,\cdots,f_n)$ where $f_i = e_{n-i+1}\forall 1\leq i\leq n$. Clearly the first columns of $J$ in this basis is given by $Jf_1 = \lambda e_n + e_{n-1} = \lambda f_1 + f_2$ which corresponds to the column matrix with first two entries being $\lambda, 1$ respectively and everything else is $0$. The $j^{\text{th}}$ column $(1\leq j<n)$ is given by $Jf_j = Je_{n+1-j} = \lambda e_{n+1-j} + e_{n-j} = \lambda f_j + f+{j+1}$ which corresponds to the columns where the $j^{\text{th}}, (j+1)^{\text{st}}$ entries are $\lambda$, $1$ respectively, and everything else is $0$. This means that $[J]_{\pmb e} = [J]_{\pmb f}^t$. Since both the matrices $[J]_{\pmb e}, [J]_{\pmb f}$ correspond to the same linear operator, but represented in different bases, they are similar. This proves that every Jordan block is similar to its transpose.
\end{sol}